%Helper file which allows me compile the article without worrying about ABNT stuff

\documentclass[12pt]{article}
\usepackage[a4paper,top=25mm,bottom=25mm,width=150mm]{geometry}
\usepackage[brazilian]{babel}
\usepackage[backend=biber,style=authoryear,sorting=nyt]{biblatex}
\usepackage{datetime}
\addbibresource{refs.bib}

\usepackage[T1]{fontenc}
\usepackage[utf8]{inputenc}

\usepackage{listings}
\usepackage{graphicx}
\lstset{language=Python,
  basicstyle=\ttfamily\scriptsize}

\newcommand{\figblocks}{
\begin{figure}
\includegraphics[width=0.6\linewidth]{figures/blocks.jpg}
\caption{Diagrama de blocos do processador SAP1. \cite{malvino}}
\label{f-blocks}
\end{figure}
}

\newcommand{\figblockmux}{
\begin{figure}
\includegraphics[width=0.5\linewidth]{figures/42-mux.png}
\caption{Bloco representando circuito multiplexador 4:2.}
\label{f-42mux}
\end{figure}
}

\newcommand{\fighmux}{
\begin{figure}
\includegraphics[width=0.5\linewidth]{figures/h-mux.jpg}
\caption{Diagrama de circuito multiplexador 4:2 construido hierarquicamente usando 3 multiplexadores 2:1.}
\label{f-hmux}
\end{figure}
}

\newcommand{\figcd}{
\begin{figure}
\includegraphics[width=\linewidth]{figures/class-diagrams.png}
\caption{Diagrama de classes para ferramenta.}
\label{f-cd}
\end{figure}
}

\newcommand{\logo}{
\includegraphics[width=0.3\textwidth]{figures/uninter-logo.png}\\
}

\newcommand{\parecer}{
\includepdf[pages=-]{figures/parecer-tecnico.pdf}
}

\renewcommand{\cite}{\parencite}

\newdateformat{mydate}{\THEMONTH \/ \THEYEAR}

\usepackage{pdfpages}

\usepackage{hyperref}
\hypersetup{
  colorlinks,
  citecolor=black,
  filecolor=black,
  linkcolor=black,
  urlcolor=black
}

%Macros 
\newcommand{\disciplina}{disciplina de computação gráfica}
\newcommand{\aluno}{Bruno Gomes}
\newcommand{\prof}{Prof. Marcos Eduardo Pivaro Monteiro}
\newcommand{\cidade}{curitiba}
\newcommand{\estado}{paraná}
\newcommand{\curso}{bacharelado em engenharia da computação}

\begin{document}

\begin{titlepage}
  \begin{center}
    \vspace{0.7cm}
    \textbf{CENTRO UNIVERSITÁRIO INTERNACIONAL UNINTER}\\
    \vspace{0.3cm}
    \textbf{CURSO DE ENGENHARIA DA COMPUTAÇÃO}\\
    \vspace{3cm}
    \logo
    \vspace{2cm}
    \textbf{\MakeUppercase{relatório de estágio supervisionado obrigatório}}\\
    \vspace{5cm}
  \end{center}
\end{titlepage}

\begin{center}
  \MakeUppercase{\cidade}\\
  \mydate \today\\
  \MakeUppercase{\aluno}\\
  \vspace{7cm}
  \textbf{RELATÓRIO DE ESTÁGIO}\\
\end{center}

\begin{minipage}[t]{0.5\textwidth}
\hfill
\end{minipage}
\begin{minipage}[t]{0.5\textwidth}
  \begin{flushright}
  Relatório de estágio apresentado ao Curso de Engenharia de Computação do Centro Universitário UNINTER.
  \end{flushright}
\end{minipage}
  

\newpage
\begin{center}
\MakeUppercase{\cidade}\\
\mydate \today \\
\end{center}
\tableofcontents

\newpage
\section{APRESENTAÇÃO DA EMPRESA}
O Grupo Educacional Uninter é o grupo fundador do Centro Universitário Internacional Uninter.
O Centro Universitário Uninter é uma instituição de ensino superior focada na modalidade de ensino a distância (EaD).
O Centro Universitário se destaca por ter recebido a nota máxima do MEC na modalidade de EaD.


\newpage
\section{HISTÓRICO DA EMPRESA}
O grupo UNINTER nasceu em 1996 e está presente até os dias de hoje.
Durante essa jornada o grupo esteve em constante inovação e mudou sua área de atuação diversas vezes.

Criado em 1996, originalmente conhecido como Instituto Brasileiro de Pós-Graduação e Extensão (IBPEX), fundado pelo professor Wilson Picler, a instituição ofertava cursos de pós-graduação presencial.
No ano de 2000, o grupo expandiu seu mercado e criou a Faculdade Internacional de Curitiba (FACINTER) e começou a ofertar cursos de graduação.
Em 2003, o grupo tomou seus primeiros passos para introduzir a modalidade EaD ao seu repertório e em 2004 foi fundada a Faculdade de Tecnologia Internacional (FATEC) para a oferta de cursos tecnológicos.

Em 2012, o grupo deu um grande salto e fundiu a FATEC e a FACINTER para criar o  Centro Universitário Internacional UNINTER.
Essa nova instituição recebeu o título de Centro Universitário o que lhe permitiu mais autonomia do MEC.

Atualmente, a UNINTER continua se expandindo e vem expandindo o número de cursos ofertados pelo grupo.

\newpage
\section{ATIVIDADES DESENVOLVIDAS}
\subsection{INTRODUÇÃO}
\subsubsection{Problema de pesquisa}
O projeto de pesquisa engloba a construção de um simulador de circuitos digitais em Python.
O simulador deve usar objetos para modelar circuitos digitais.
A representação de um circuito é feita por classes escritas pelo usuário.
Um requisito é que o simulador deve ser capaz de simular um processador.
Para atender a esse requisito foi construído um processador virtual que implementa a arquitetura SAP1.

\subsubsection{Objetivos}
a) Objetivo Geral:\\
Construir uma ferramenta para simulação de circuitos digitais e simular um processador na ferramenta criada.\\
b) Objetivos Específicos:
\begin{itemize}
\item Definir características do simulador e delimitar escopo do projeto.
\item Modelar circuitos digitais como objetos.
\item Codificar simulador.
\item Construir processador.
\end{itemize}

\subsubsection{Justificativa}

A disciplina de circuitos digitais é a base que explica o funcionamento dos sistemas microprocessados atuais.
Construir um circuito digital é aplicar os conhecimentos teóricos de maneira prática, um ato fundamental para o aprendizado.
Para construir circuitos digitais existem três maneiras bem estabelecidas: construir o circuito físico utilizando chipes fabricados, construir o circuito em uma ferramenta de simulação gráfica e construir o circuito em uma ferramenta HDL.
Será explorado as vantagens e desvantagens de cada método e em seguida será proposto uma alternativa que utiliza a orientação a objetos para construir circuitos.

Os circuitos físicos são normalmente construídos em protoboards utilizando chipes e jumpers.
Cada chipe é um circuito digital encapsulado que realiza uma operação imutável.
Existem um arsenal de chipes que realizam operações diferentes.
Embora seja o método clássico, a medida que os circuitos crescem em complexidade, as chaces de cometer um erro aumentam devido ao número de conexões que devem ser feitas.
Outra desvantagem é o tempo necessário para construir um circuito físico.
O processo não é eficiente, demandando muito tempo para verificar conexões e procurar por informações em datasheets.
Outro problema refere-se à dificuldade de abstrair e modularizar um circuito, caso ele não exista pré fabricado será obrigação do usuário de construir esse circuito sempre que necessário, o que é ineficiente.
A vantagem desse método é que ele é usado na fase de prototipação de dispositivos eletrônicos.
Devido a essas características muitas vezes são adotadas ferramentas de simulação ao invés de circuitos físicos.

Existem uma enorme quantidade de ferramentas para simulação porém nem todas são apropriadas para todos os públicos.
No geral, as ferramentas de simulação mais simples fazem uso de diagramas de conexão.
Nesses diagramas, o usuário deve posicionar os blocos correspondentes aos circuitos e traçar as conexões entre eles.
Essa metodologia é pouco melhor do que os circuitos físicos, sendo repetitiva e possuindo baixa abstração.
Embora existam inúmeras ferramentas de simulação gráficas, o foco principal será nas HDL (linguagem de descrição de hardware).

As HDL são similares a linguagens de programação e são usadas para descrever circuitos.
A descrição de circuitos em HDL é feita por arquivos de texto, análogos a programas de computador.
Uma vez escritos, os circuitos podem ser usados para simulação ou síntese, sendo uma solução completa para o processo de design.
Segundo \textcite{harris}, as HDLs foram criadas na década de 90 para abstrair o processo de design dos engenheiros responsáveis pelos projetos de circuitos digitais, pois elas realizam o processo de otimização e simplificação do circuito.
Devido a isso, as HDLs agilizam o processo de design e poupam tempo do engenheiro.
O inconveniente dessas ferramentas é que elas foram projetadas para suprir a necessidade dos engenheiros projetando circuitos digitais para o mercado e, não para a simulação de circuitos para fins didáticos.
A consequência disso é que as HDL possuem mais funcionalidades que necessário para um contato inicial e por terem sido projetadas para a indústria, a representação de circuitos não é necessariamente intuitiva.

Deseja-se criar uma ferramenta que possui características similares às HDL, porém que apliquem um modelo mais simples e consequentemente mais limitado.
Essa nova ferramenta seria, idealmente, simples, a fim de ser usada por iniciantes em circuitos digitais.
Ela seria uma alternativa para estudantes de circuitos digitais que desejam construir um processador sem ter que batalhar com a ferramenta ou com o ambiente de desenvolvimento.

Após a construção da ferramenta, ela será usada para simular um processador que implementa a arquitetura SAP1.
A ferramenta deve ser capaz de cumprir essa tarefa pois, segundo \textcite{harris} a construção de um processador é quase um rito de passagem para qualquer engenheiro da computação.
Então, deseja-se que essa ferramenta seja capaz de suprir essa necessidade quase que universal dos estudantes de engenharia.

\subsection{REVISÃO DE LITERATURA}
\subsubsection{Características do simulador}
Primeiramente serão definidas as características do simulador, o que também define o escopo do projeto.

O simulador construído faz uso do conceito de hierarquia para construir circuitos.
Segundo \textcite{harris}, hierarquia é um princípio usado para lidar com a complexidade de circuitos digitais, dividindo um sistema em módulos e submódulos até que cada parte seja fácil de entender.
Ainda segundo \textcite{harris}, esse mesmo princípio é usado em linguagens HDL para descrever um circuito de maneira estrutural.

Os circuitos digitais têm uma forte correlação com a lógica matemática.
Essa relação é graças ao conceito de \emph{static discipline}, que restringe os valores de voltagem que um circuito interpreta a duas possibilidades \cite{kaufman}.
Sendo assim, os circuitos digitais fazem uso dos operadores lógicos AND, OR, XOR e NOT.
Esses operadores têm os seus equivalentes na forma de portas lógicas na eletrônica digital.
O simulador disponibiliza as portas AND, OR e XOR como parte da biblioteca.
As portas lógicas são os átomos da eletrônica digital e circuitos mais complexos são implementados a partir delas usando o conceito de hierarquia.

Circuitos digitais reais possuem atrasos e são sujeitos a ruídos \cite{kaufman}.
Esses fatores são de grande preocupação para a construção de circuitos digitais reais e ferramentas HDL levam em consideração essas variáveis.
O simulador ignora essas condições e trabalha com sinais digitais perfeitos (sem ruído) e instantâneos (sem atraso).
Essa simplificação possibilita uma ferramenta menos complexa e mais amigável.

Uma grande vantagem da HDL é que ela pode ser usada para simulação e síntese de circuitos.
O processo de síntese da HDL realiza simplificação de portas lógicas e realiza otimizações para o tipo de tecnologia que será usada no processo de fabricação \cite{harris}.
A ferramenta construída não realiza o processo de síntese ou de simplificação de portas lógicas.

O simulador faz uso da orientação a objetos para construir circuitos e foi definido um modelo que descreve circuitos usando classes.
Para criar um circuito o usuário deve escrever uma classe e especificar os sub-circuitos e as ligações entre eles.
As classes criadas podem então ser usadas para a criação de outro circuitos, fazendo uso do conceito de hierarquia.

O processo de simulação será feito através de scripts ou via um interpretador.
A interação via interpretador é mais simples porém limitada e repetitiva.
A vantagem é que o interpretador é um ambiente interativo com feedback instantâneo que ajuda no processo criativo.
Os scripts consistem de: inicializar o circuito a ser simulado e variar os sinais de entrada desse circuito.
Assim, o usuário pode usar um script para gerar a tabela verdade do circuito que ele criou e verificar se o mesmo esta correto.
O uso de scripts pode parecer desnecessariamente complicado porém apresenta a vantagem de ser automatizável, sendo possível automatizar o processo de teste e simulação de circuitos usando funções. 

\subsubsection{Eletrônica digital em objetos}

O objetivo dessa seção é: apresentar o paradigma orientado a objetos, identificar os constituintes da lógica digital e definir quais objetos farão parte do modelo que irá ser usado pela ferramenta. 

A programação orientada a objetos parte do princípio de criar um modelo, em software, um sistema físico.
Esse paradigma é explicado por \cite{objects} como:

\begin{quote}
O paradigma de orientação a objetos é baseado em uma correlação intuitiva entre um software simulando um sistemas físico e o sistema físico propriamente dito.
É feita uma analogia entre construir um modelo algorítmico de um sistema físico a partir de componentes de software e construir um modelo mecânico do sistema a partir de objetos concretos.
Sendo assim, por analogia, os próprios componentes de software passam a ser chamados de objetos.
\end{quote}

A orientação a objetos facilita a macro visão, pois o mesmo passa a ser mais concreto e modular.
Deseja-se criar um modelo orientado a objetos para circuitos digitais que seja intuitivo para o usuário.
Para construir um modelo orientado a objeto é necessário identificar o que seriam os objetos que regem a lógica digital.

A literatura define um circuito, no escopo da eletrônica digital, como sendo "[...] uma rede de processamento de variáveis com valores discretos.", \cite{harris}.
Como um circuito é uma rede isso implica que exista comunicação entre os elementos dessa rede.
Em circuitos digital, tal como em circuitos analógicos, os fios são os objetos que transmitem informação entre os elementos.
Essa definição expõe necessidade da existência dos objetos fio e circuito na eletrônica digital, ambos farão parte do modelo de software da ferramenta.

Como visto, fios são os propagadores de informação.
Cada fio contém uma unidade de informação, chamado de bit.
Alguns circuitos recebem múltiplos bits de informação, tal como um somador de 8 bits.
Naturalmente o aumento do número de bits sendo transmitido aumenta o número de fios.
É desejável abstrair essa característica.
Na literatura, um grupo de fios carregando partes da mesma informação para um circuito são agrupados e denominados de Bus.
Como um fio transmite um bit uma bus transmite vários bits.
Um grupo de bits será denominado de sinal.
Isso introduz dois novos objetos: bus e sinal.

O objeto sinal é um pouco peculiar por não ter um equivalente preciso em um circuito físico, vendo que pode-se apenas medir o sinal (voltagem) de apenas um fio de cada vez.
O objeto sinal é um exemplo de uma abstração definida a fim de simplificar o modelo.
Para o usuário, é mais conveniente representar uma Bus como um único número, ao invés de um conjunto de bits.
O sinal é construído a partir desse conjunto de bits que são interpretados como um número binário.
De maneira geral, o sinal de uma bus com $n$ fios é um número natural no intervalo $[0,2^n -1]$, onde $n \in N $ \cite{harris}.

Foi visto o que um circuito faz mas não sua interface.
Segundo \textcite{harris}, o circuito é como uma caixa preta que contém terminais de entrada, terminais de saída, uma função que computa a saída a partir das entradas e uma especificação de atrasos no circuito.
Internamente um circuito contém fios e elementos, onde elemento se refere à outros circuitos \cite{harris}.
Essa definição introduz um novo elemento, o objeto terminal.
Pode-se considerar os terminais de entrada e saída como sendo a interface de um circuito, pois definem como interagir com o mesmo.
Essa definição esta de acordo com \textcite{kaufman}, onde é dito que "[...] o acesso à circuitos eletrônicos é feito por terminais.".
A partir das definições acima pode-se entender que terminais separam a parte interna e externa de um circuito, em termos de classes, podemos dizer que circuitos contém objetos terminais e que terminais estão associados a um conjunto de fios, ou seja uma Bus.

Em conclusão, circuitos digitais são compostos por: fios, terminais e circuitos.
Cada fio armazena uma unidade de informação, um bit, que é transmitida a outros circuitos através de terminais neles.
Uma abstração que ocorre naturalmente na disciplina é agrupar um conjunto de fios e denominá-los de Bus, buses possuem um sinal.
Os objetos que devem ser codificados então são: fio, bus, terminal e circuito.

\subsubsection{Construção do simulador}

O processo de construção do simulador envolve traduzir as relações descritas na seção anterior para a linguagem de software.
Esse processo formalmente é feito usando diagramas UML.
A fígura \ref{f-cd} formaliza o modelo criado tal que ele possa ser transformado em software.
A fígura especifica a relação entre as classes do software usando a notação tradicional das linguagens UML.
Dentre elas: Buses possui Wire, Signal é associado a Bus, Terminal é associado a Bus e Circuit possui Terminal.

\figcd

Embora tenha sido definido um modelo de software para circuitos digitais, esse modelo está incompleto.
Como foi visto, circuitos digitais são uma rede de processamento de variáveis.
Ao modificar a entrada de um circuito, isso modifica a saída do mesmo.
No caso da saída estar conectada, existe a necessidade de atualizar o circuito encadeado.
Uma solução para isso seria atualizar todos os objetos circuitos cada vez que um sinal seja alterado, mas isso é um enorme desperdício.
Sendo assim, é necessário um componente em software que atualize somente os circuitos afetados.

O simulador faz uso de eventos e um objeto auxiliar para orquestrar o processo de atualização de circuitos.
Essa solução é um padrão de projeto conhecido como \emph{Observer}.
O intuito desse padrão é "definir uma dependência de um-para-muitos entre objetos tal que quando um objeto mude de estado, todo seus dependentes sejam notificados e atualizados automaticamente." \cite{gof}. 

\fighmux

Contextualizando, suponha o circuito da fígura \ref{f-hmux}, um circuito multiplexador 4:2 feito a partir de 3 multiplexadores 2:1.
Nota-se que o circuito possui 6 entradas e 1 saída.
Suponha que o sinal no terminal D0 seja modificado, a bus D0 está ligada somente ao circuito M1 e nenhum outro, seria um enorme desperdício se o simulador recalculasse a saída de todos os circuitos do programa.
Fica aparente a necessidade de uma entidade que controle quais circuitos devem ser atualizados a cada alteração de sinal e foi nesse sentido que foi utilizado o padrão Observer.


\subsubsection{Processador SAP1}
\paragraph{Arquitetura SAP1}
Parte do projeto envolve a construção de um processador.
Existem inúmeras arquitetura didáticas que poderiam ser implementadas porém foi escolhido a arquitetura SAP1.
Nessa seção será apresentado os aspectos principais dessa arquitetura.

A arquitetura SAP1 é uma arquitetura de 8 bits com um conjunto de instruções mínimo.
Essa arquitetura é apresentada por \cite{malvino}.
Embora contenha um conjunto de instruções reduzido ela apresenta os elementos comuns a todas arquitetura, tais como registradores, unidade de controle, ALU, MAR e etc.
Essa arquitetura foi escolhida por ter um balanço entre um conjunto de instruções simplificado e microarquitetura moderadamente realística.

A fígura \ref{f-blocks} apresenta um diagrama de blocos da arquitetura.
Notam-se dois registradores conectados ao somador (A e B), um registrador de saída (O), banco de memória de 16x8 e os demais componentes esperados como registrador de endereço de memória (MAR), contador de programa (PC) e unidade de controle (CU).
Essa arquitetura apresenta uma única bus (W) que é usada para transmitir endereços e dados, para que não haja sobreposição de valores em W a arquitetura faz uso de lógica de três estados.

\figblocks

O conjunto de instruções da arquitetura é apresentado de maneira resumida na tabela \ref{t-inst}.
Essas instruções são: ADD, LDA , SUB e HLT, porém a instrução HLT foi omitida na construção do processador por ser redundante.
É importante destacar que essa arquitetura não apresenta nenhuma instrução de jump, o que limita muito o tipo de computação que pode ser feito.
Todas as instruções dessa arquitetura seguem o mesmo formato, 8 bits de comprimento sendo os bits $[0,3]$ destinados à endereço ou dado e os bits $[4,7]$ sendo reservados para o OP code da instrução.
Em seguida serão abordado os funcionamentos de cada instrução.

\begin{table}
  \caption{Instruções da arquitetura SAP1 e suas descrições.}
\begin{tabular*}{0.5\linewidth}{lcp{4cm}}
  \hline
 Inst. & Op code & Descrição \\
 \hline
 LDA         &     0x0 & Carrega o acumulador com o valor contido nos bits [0,3] \\
 ADD         &     0x1 & Soma o valor presente no acumulador com o valor contido no endereço apontado pelos bits [0,3] da instrução. \\
 SUB         &     0x2 & Subtrai o valor presente no acumulador com o valor contido no endereço apontado pelos bits [0,3] da intrução.\\
 OUT         &     0xE & Escreve o valor contido no acumulador ao registro de saída (O). \\ 
 \hline
\end{tabular*}
\label{t-inst}
\end{table}


A instrução LDA é utilizada para carregar um valor ao registrador A.
Essa instrução contém nos 4 primeiros bits um endereço que será acessado pelo MAR e o valor contido na memória é carregado no registrador A.
Um exemplo da instrução LDA em linguagem de máquina é 0x19, ou seja, carregar o valor contido no endereço 9 no registrador A.

As instruções ADD e SUB possuem um funcionamento análogo porém realizam operações matemáticas diferentes, essas instruções somam e subtraem, respectivamente, o valor armazenado no registrador A ao valor carregado no registrador B.
Elas possuem em seus 4 primeiros bits um endereço de memória cujo valor é armazenado é copiado ao registrador B.
A ALU então opera sobre os registradores A e B e o resultado dessa operação é armazenado no registrador A.
Um exemplo dessa instrução em linguagem de máquina seria 0x25, ou seja, carregar o valor no endereço 5 ao registrador B, somar os valores de A e B e armazenar o resultado em A. 

A instrução OUT é extremamente simples, ela copia o valor contido no registrador A para o registrador O.

O processador possui 3 registradores disponível ao usuário: registrador A, registrador B e registrador O.
O registrador A funciona como acumulador e após as instruções de soma (SUM) e subtração (SUB) o resultado da operação é armazenado nele; é possível também alterar diretamente o valor de A utilizando a instrução LDA que carrega o valor contido no endereço encodado na instrução a ele.
O registrador B funciona como um registrador auxiliar para as instruções de SUM e SUB.
O endereço indicado pelas instruções é acessado e o valor contido nele é carregado ao registrador B.
O valor do registrador B é alterado a cada instrução de SUM ou SUB.
Finalmente, o registrador O é utilizado para apresentar informação ao usuário.

\paragraph{Micro arquitetura}
Uma importante característica da micro arquitetura do SAP1 é que ela possui micro instruções.
A micro arquitetura construída foi fortemente baseada na implementação feita na literatura.
A característica principal dessa microarquitetura é que todas as instruções são executadas em 6 pulsos do clock.
\footnote{Os detalhes dos ciclos das micro instruções são desnecessários para a construção do processador e para a compreensão da arquitetura da perspectiva do programador. A explicação completa do funcionamento dessa arquitetura pode ser encontrado na obra citada.}

As conexões entre os componentes do processador podem ser visualizados na imagem \ref{f-blocks}.
Como mencionado anteriormente, essa arquitetura usa lógica de três estados para preservar o valor escrito em W.
Os blocos PC, RAM, IR e A apresentam lógica de três estados na saída e os blocos A, B, O, IR e MAR apresentam lógica de três estados nos terminais de entrada.
O controle dos blocos é regida pelas 12 flags presentes na imagem.
A tabela \ref{t-flags} apresenta o propósito dessas flags. 

\begin{table}
  \caption{Descrição das flags de controle do processador.}
\begin{tabular}{ll}
\hline
Flag & Descricao \\
\hline
cp & Incrementar PC no proximo clk \\
ep & Tristate de saida do PC \\
lm & Tristate de entrada do MAR \\
ce & Tristate saida da RAM \\
li & Tristate de entrada do IR \\
ei & Tristate de saida do IR \\
la & Tristate de entrada do acumulador \\
ea & Tristate de saida do acumulador \\
su & Flag de subtracao para o adder/subtractor \\
eu & Tristate de saida para o adder/subtractor \\
lb & Tristate de entrada para o registrador B \\
lo & Tristate de entrada para o registrador output \\
\hline
\end{tabular}
\label{t-flags}
\end{table}


A literatura apresenta duas alternativas para a construção da unidade de controle (CU), uma baseada em lógica combinacional e a outra em uma memória ROM.
Foi escolhido a CU que utiliza memória ROM por ser mais simples de implementar no simulador.
Essa CU possui dois terminais de entrada, IW (palavra de instrução) e MIW (palavra de micro instrução) e um de saída CW (palavra de controle).
A entrada IW carrega os 4 bits que representam o opcode da instrução sendo executada e a entrada MIW contém o ciclo de micro instrução (número de 0 a 5 gerado pelo MIC).
A saída CW é uma bus de comprimento 12 que orquestra os blocos do processador, essas flags são utilizadas para controlar o comportamento dos registradores, CU e lógica de três estados.
Como explicado acima, essa CU é uma lookup table para produzir as flags das micro instruções.
O endereço da memória é dado pela concatenação das buses IW e MIW.

\subsection{METODOLOGIA}
\subsubsection{Desenvolvimento PDD}
\paragraph{Representação}
A representação de um circuito é de extrema importância para o simulador.
Uma representação complexa e detalhada torna a ferramenta menos acessível a novos usuários.
Sendo assim, a representação escolhida tentou ser simples e familiar, pois modela circuitos de maneira análoga ao apresentado nas literaturas.

Definir a representação foi o primeiro passo do processo criativo pois ela define a interface do código.
A interface definida passou por uma série de iterações até chegar em um resultado que o autor considerou agradável e compreensível.
Como um dos objetivos é desenvolver uma ferramenta amigável, primeiramente será demonstrado snippets de código do modelo implementado.

O código abaixo cria uma porta do tipo AND e altera os sinais de entrada dessa porta.
Percebe-se que a porta é criada usando a notação usual de construção de objetos usada nas linguagens orientada a objetos.
A porta AND possui duas entradas (A e B) e uma saída (Y), e como foi visto anteriormente cada uma dessas conexões é um terminal.
A interação com os terminais é feita a partir do operador ponto (usado para acessar atributos).
No código acessamos a bus ligada ao terminal "A" da porta lógica e em seguida modificamos o valor do sinal dela.
Essa operação foi executada de maneira enxuta e legível, sem dificuldades para o usuário.
O mesmo procedimento é executado na porta "B" e finalmente executa-se a função print para verificar o sinal no terminal "Y".
Também é possível chamar print na porta como um todo e é impresso todas as informações referentes ao circuito.

\begin{lstlisting}
  porta = AND() #cria porta do tipo AND e armazena a referencia na variavel porta
  porta.a.signal = 1 #modifica o sinal presente na Bus conectada ao terminal A
  porta.b = 1
  print(porta.y) #imprime informacao do terminal Y
  print(porta) #impreme informacao do circuito
\end{lstlisting}

A composição de circuitos também é feita a partir de operações nos objetos.
O código abaixo cria duas portas AND, p1 e p2, e conecta a saída de p1 à entrada de p2.
A conexão dos circuitos é feita a partir do método "connect".
Esse método faz uso de argumentos do tipo chave-valor.
O método recebe um conjunto chave-valor onde a chave é o nome de um terminal do circuito e o valor é a bus que será ligada ao terminal.
A conexão entre os dois circuitos foi feita em uma linha de código de maneira clara e simples.

\begin{lstlisting}
  p1, p2 = AND(), AND()
  p2.connect(a=p1.y) #Liga saida de p1 na entrada 'a' de p2
\end{lstlisting}

O próximo snippet utiliza os conceitos apresentados até agora e aplica modularidade ao processo.
O código a seguir define um novo circuito, ou seja uma classe, e a ele damos o nome "NovoCircuito".
NovoCircuito herda de uma classe abstrata disponibilizada pela ferramenta e implementa o mesmo circuito do exemplo anterior (duas portas AND conectadas).
A vantagem de criar uma classe para esse circuito é que ele fica reutilizável e basta criar uma nova instância do objeto, além de aplicar o conceito de hierarquia, ou seja, construímos um circuito mais complexo a partir de circuitos simples.
A classe herdeira deve definir os nomes dos terminais de entrada e saída, e implementar o método "make".
Os atributos "input\_labels" e "output\_labels" devem ser uma lista de strings que definem os nomes dos terminais de entrada e saída, respectivamente.
O método make é aonde as conexões entre sub circuitos são definidas.
Algo de grande importância no método make é que o usuário deve chamar o método \texttt{self.get\_inputs()}.
Esse método retorna um struct contendo as buses internas ao circuito sendo construído.
Esse passo é essencial, pois é a partir dessas buses que é feita a conexão do NovoCircuito com os subcircuitos presentes nele.
Sem os objetos retornados pelo \texttt{get\_inputs}, NovoCircuito seria uma caixa preta com entradas ligadas a nenhum subcircuito.

\begin{lstlisting}
class NovoCircuito(BaseCircuit):
    input_labels = ['a', 'b', 'c'] #lista de nomes dos terminais de entrada
    output_labels = ['y'] #lista de nomes dos terminais de saida

    def make(self):
        i = self.get_inputs() #objeto que contem as buses internas ao circuito NovoCircuito.
        and_1 = AND()
        and_1.connect(a=i.a, b=i.b) #Conecta as buses de entrada a e b ao circuito and_1
        and_2 = AND(a=and_1.y, b=i.c) 
        self.set_outputs(y=and_2.y) #Conecta a saida do circuito and_2 a saida 'Y' do NovoCircuito
\end{lstlisting}

Finalmente, o último snippet demonstra a utilização de NovoCircuito.
O processo é idêntico à criação de uma porta AND.

\begin{lstlisting}
  novo = NovoCircuito()
  novo.a.signal = 1
  novo.c.signal = 0
  print(novo)
\end{lstlisting}

Os quatro snippets acima são curtos porém eles, de certa forma, são tudo o que um usuário precisa entender sobre a ferramenta.
Os snippets introduzem a "linguagem" da ferramenta, com essa linguagem o usuário pode modelar qualquer circuito que ele queira.
A simples representação escolhida é o diferencial da ferramenta e a partir dela, pode-se criar circuitos de complexidade incremental, usando modularidade e hierarquia, e tudo isso com uma interface de código consistente.

\paragraph{Arquitetura do simulador}
As classes usadas para definir o modelo foram introduzidas no referêncial teórico.
Nessa seção serão apresentadas as interfaces e as funcionalidades dessas classes.

A classe Wire, modela um fio que armazena um bit.
Wire armazena esse bit em uma variável booleana que pode ser modificado usando os metodos \texttt{set()} e \texttt{reset()}.
Objetos do tipo Wire são imutáveis e essa característica vai ser fundamental para a implementação da lógica de atualização do simulador.

A classe Bus representa um conjunto de fios (objetos do tipo Wire).
Essa classe define uma abstração em cima da classe Wire, fazendo com que Wire não seja relevante ao usuário.
Ao invés de armazenar 1 bit de informação, Bus armazena n bits, onde n é o número de fios que constituem a mesma.
Sendo assim, Bus possui o atributo sinal que representa um número construído a partir dos bits armazenados nos fios.
Bus possui o atributo wires que é uma lista de objetos wire.
Como a bus representa uma informação multibit a ordenação dos fios na lista é importante.
O fio na posição $n$ na lista representa o bit na posição $n$.
Objetos do tipo Bus são mutáveis, um usuário pode realizar operações em Bus para construir outras Bus a partir da mesma.
Por exemplo, suponha que uma Bus W tenha 8 wires, o usuário pode construir uma Bus Y que tenha os 4 primeiros wires de W.
Isso foi feito para facilitar o acesso a bits em uma Bus, sendo esse um caso de uso comum na construção de circuitos digitais.
Finalmente, para alterar o valor de um sinal na Bus basta usar o comando bus.signal = 3.

A classe Terminal é usada para definir entradas e saídas em um circuito.
O objeto Terminal pode ser visualizado como uma porta lógica do tipo Buffer onde existe uma entrada e uma saída, porém Terminal possui algumas conveniências.
O terminal pode funcionar como um inversor e um buffer de três estados, basta configurá-lo corretamente na fase de construção do objeto.
Tendo um único objeto que apresente todas essas funcionalidades é desejável pois fica extremamente fácil de reutilizar circuitos.
Suponha que tenha sido construído um circuito multiplexador comum e que em um momento futuro o usuário precise inverter uma das entradas do multiplexador ou adicionar um buffer de três estados na saída.
Assim, as conveniências da classe Terminal evita a necessidade de construir uma outra classe ou adicionar uma porta NOT no circuito, basta informar os parâmetros corretos na hora de inicializar o objeto.
Os atributos de Terminal são "bus-in" e "bus-out", que definem as buses de entrada e saída, respectivamente, "en" que define a configuração do buffer de três estados, "invert" que configura o terminal como uma porta NOT e "size" que define o tamanho das buses que podem ser conectados ao terminal.
O método propagate replica o sinal da bus de entrada para a bus de saída, dependendo da configuração do terminal.
Por exemplo, caso o comportamento de três estados do terminal esteja com impedância alta na saída, o método propagate não mudará o valor da bus de saída;
ou caso o terminal esteja configurado para negar a saída, o sinal de saída será o completo do sinal de entrada.

Finalmente, temos a classe que modela circuitos, a classe BaseCircuit.
A classe BaseCircuit define uma classe abstrata.
Essa classe implementa um construtor que, a partir dos atributos "input\_labels" e "output\_labels", automaticamente cria terminais de entrada e saída, buses para esses terminais e chama o método \texttt{make()}, responsável pela criação e configuração dos subcircuitos.
Fora o método make, essa classe implementa o método update, que é responsável por atualizar o circuito, e nesse ponto existe uma sutileza do design da ferramenta.

Deve-se lembrar que o simulador faz uso da  hierarquia, ou seja, circuitos são construídos a partir da combinação de circuitos mais simples.
Isso faz com que a atualização do circuito seja, na verdade, a atualização dos subcircuitos que o compõe. 
Como os circuitos são construídos a partir das portas lógicas, isso faz com que todo o trabalho de atualização seja feito a um nível de AND, OR e XOR.
Sendo assim, para implementar o método de atualização da classe BaseCircuito, o circuito deve chamar o método propagate para todos os terminais de entrada e chamar o método de atualização nos subcircuitos.
Esse ciclo de transferência de responsabilidades se repete até chegar nas portas lógicas, onde são realizados as operações da álgebra booleana sobre os sinais de entrada.

Agora que foram explicadas todas as relações entre as classes do modelo, o único passo que falta é a implementação das portas lógicas.
Como toda a complexidade de criar uma interface constante esta definido na classe BaseCircuit, a implementação das portas lógicas é feita de maneira similar aos exemplos dados na seção anterior.
A única diferençã é que as portas lógicas dão override no método update, pois como visto acima, elas têm uma responsabilidade diferente aos outros circuitos.

O snippet abaixo contém o código usado para implementar a porta AND.
Nota-se que o código faz uso de herança e aproveita os métodos ja implementados na classe pai para atualizar as buses internas do circuito.
A operação lógica AND é feita usando o operador "\&" presente por padrão na linguagem Python.
As portas lógicas OR e XOR foram construidas de maneira análoga, substituindo o operador \& pelos operadores $|$ e \^{}, para a porta OR e XOR, respectivamente

\begin{lstlisting}
class AND(BaseCircuit):
    """
    Implementacao da porta AND.
    Entradas: a e b
    Saidas: y
    Classe realiza override do metodo update e realiza operacoes binarias sobre os sinais das buses
    """
    input_labels = 'a b'.split()
    output_labels = 'y'.split()
    def update(self):
        super().update() #chama o metodo update na classe pai, propaga o sinais de entrada para as buses internas
        i = self.get_inputs() #recebe buses internas
        output_signal = i.a.signal & i.b.signal #realiza a operacao binaria and sobre os sinais de entrada
        i.y.signal = output_signal #altera o sinal da bus de saida
        super().update() #novamente chama o metodo update para atualizar o sinal da bus de saida
\end{lstlisting}

Foi feita uma descrição alto nível das classes, sem entrar em particularidades de implementação na linguagem Python.
As classes como foram apresentadas acima definem o comportamento do modelo do simulador.
O código fonte completo de todas as classes, como foi explicado acima está disponível em http://github.com/lodek/pdd.

\paragraph{Back end}
O único elemento que resta para o funcionamento do simulador é uma entidade que automatize o processo de atualização de circuitos.
Em um circuito eletrônico digital real, quando o valor entrada de um circuito muda, após o atraso propagacional do mesmo, o valor de saída muda.
Esse comportamento foi usado como inspiração para solucionar o problema de atualização do processador e a solução foi implementada o padrão Observer.

Como foi visto, o Observer é usado quando existe uma dependência entre objetos.
O uso previsto do Observer é de uma relação um para muitos, porém o caso de uso no simulador é uma relação de muitos para muitos.
Para solucionar o problema acima, foi definido que um circuito depende de um conjunto de fios (objetos Wire).
O padrão Observer então permite que cada vez que um Wire seja modificado, esse informe aos seus dependentes e os atualize de maneira automática.
Nesse caso, foi definido que ao alterar o valor de entrada de qualquer circuito, ou seja, os valores dos objetos Wire, isso gera um evento.

A implementação do padrão Observer foi feita a partir das classes "Updater" e "Event".
A classe Event é um objeto simples que contém apenas o atributo source, uma referência para o Wire que criou o evento.
A classe Updater possui os atributos relations e events, e possui o método update.
O atributo relations é uma hashtable onde a chave é um objeto do tipo Wire e o valor é uma lista de objetos do tipo Circuit que são os dependentes do objeto Wire.
Já o atributo events é uma lista do tipo FIFO de objeto Event.
O método update inicia o ciclo de atualização do simulador, uma versão simplificada do algorítimo de atualização é apresentada a seguir.
\begin{lstlisting}
while updater.events: #iterar sobre fifo de eventos
    eventos = updater.events.pop() #remover primeiro evento
    dependentes = updater.relations[event.wire] #lista de circuitos dependentes do objeto wire
    for dependente in dependentes: #atualizar dependentes
        dependente.update()
\end{lstlisting}
O algorítimo de atualização foi escrito em 6 linhas.
Essa implementação simples é graças ao uso de eventos e do padrão Observer que separa a responsabilidade do processo de atualização entre as classes.

Aplicando o algoritmo acima para o circuito multiplexador (fig. \ref{f-hmux}), suponha que foi modificado o sinal de D0.
Por construção, A0 notifica o Updater com um evento.
Em seguida, é iniciado o processo de atualização usando updater.update().
O Observer irá identificar o evento gerado por D0 e então verifica os dependentes de D0.
Como M1 é dependente de D0, o Observer atualiza M1, que por sua vez modifica Y1, que por sua vez notifica o Observer.
Novamente, o Observer identifica um evento presente no FIFO e verifica os dependentes de Y1.
Esse processo se repete até que o FIFO de eventos esteja vazio.
Percebe-se que a alteração de uma Bus pode iniciar um efeito cascata, fazendo com que uma série de circuitos sejam atualizados.
O uso do Observer permite que esse processo de atualização seja feito de maneira simples.

Alguns pontos são importantes a respeito desta implementação.
O cadastro dos fios no Observer é feito pela classe BaseCircuit durante a construção do objeto, tal que o usuário não precisa conhecer o backend.
Também é importante ressaltar que o Observer usa objetos do tipo Wire, e não Bus, como fontes de eventos pois os objetos Bus são mutáveis, o que impossibilita o armazenamento das relações entre circuitos.

\paragraph{Front end}
Foram definidas duas maneiras de interação com o simulador.
Uma das maneiras é um ambiente tipo sandbox em que o usuário têm uma interação dinâmica com a ferramenta usando um REPL (Read-Evaluate-Print-Loop).
Como Python é uma linguagem interpretada, a própria linguagem oferece o REPL para ser usado.
A outra interface prevista é que a simulação dos circuitos será feita a partir de scripts escritos pelo usuário.
Essa metodologia é um pouco menos intuitiva porém é muito mais eficiente, já que uma vez criado o script, basta executá-lo novamente para recriar a simulação.

Para fazer uso do ambiente sandbox, o computador deve estar com Python instalado e a biblioteca escrita.
Sendo atendido esses requisitos de ambiente, basta que o usuário inicie o interpretador Python, importe a biblioteca e chame o método de configuração inicial.
Sendo feito essa configuração inicial o usuário pode criar e interagir com circuitos como foi descrito na seção de interface.
O snippet abaixo mostra os imports que devem ser feitos e o simulador funcionando para uma porta AND.

\begin{lstlisting}
Python 3.7.4 (default, Jul 16 2019, 07:12:58) 
Type 'copyright', 'credits' or 'license' for more information
IPython 7.6.1 -- An enhanced Interactive Python. Type '?' for help.

In [1]: import pdd

In [2]: from pdd.combinational_blocks import *

In [4]: pdd.core.Wire.auto_update = True

In [24]: a = AND()

In [25]: a
Out[25]: Gate AND: a=0x0; b=0x0; y=0x0; 

In [26]: a.a.set()

In [27]: a
Out[27]: Gate AND: a=0x1; b=0x0; y=0x0; 

In [28]: a.b.set()

In [29]: a
Out[29]: Gate AND: a=0x1; b=0x1; y=0x1; 
\end{lstlisting}

A outra possibilidade de uso é através de scripts.
O uso de scripts garante que nada seja perdido, todas as instruções sempre estarão salvas nele.
Por exemplo, um usuário pode definir um módulo Python com circuitos escritos por ele e uma função que realize a manipulação dos sinais de maneira automatizada.
Isso permite que, em caso de erros na implementação do circuito, basta abrir o script, corrigi-lo e executá-lo novamente.
Um uso mais avançado, porém possível, é usar uma metodologia tipo TDD (Test Driven Development) para desenvolver um circuito.

Como exemplo de script que pode ser escrito, o snippet abaixo é um script que implementa um circuito e apresenta os valores impressos no terminal.
O script implementa um circuito multiplexador de 2 entrada como é apresentado em \cite{harris}.
Nota-se que o circuito "d0\_and" faz uso da funcionalidade inversora da classe terminal para evitar o uso de uma porta NOT.
Apos a definição do circuito, cria-se um objeto do tipo SimpleMux e modifica-se os sinais de entrada do circuito.
O snippet contém também os valores de saída apresentados no terminal.

\begin{lstlisting}
import pdd
from pdd.combinational_blocks import *
from pdd.dl import BaseCircuit
pdd.core.Wire.auto_update = True

class SimpleMux(BaseCircuit):
    """
    Circuito multiplexer com 2 entrada e 1 linha de select
    """
    input_labels = 'd0 d1 s'.split()
    output_labels = 'y'.split()

    def make(self):
        i = self.get_inputs()
        d0_and = AND(a=i.d0, b=i.s, bubbles=['b']) #AND com entrada B invertida
        d1_and = AND(a=i.d1, b=i.s)
        select_or = OR(a=d0_and.y, b=d1_and.y)
        self.set_outputs(y=select_or.y)

mux = SimpleMux()
print(mux)
mux.d0 = 1
print(mux)
mux.s = 1
print(mux)

#######################################
#Output do terminal
#######################################

~/projects/pdd $ python mux_script.py
SimpleMux: {'d0': '0x0', 'd1': '0x0', 's': '0x0', 'y': '0x0'}
SimpleMux: {'d0': '0x1', 'd1': '0x0', 's': '0x0', 'y': '0x1'}
SimpleMux: {'d0': '0x1', 'd1': '0x0', 's': '0x1', 'y': '0x0'}
\end{lstlisting}

Isso encerra a definição e construção do simulador.
Como visto acima, foi implementado um circuito multiplexador construído a partir de portas lógicas.
Partindo das portas AND, OR e XOR é possível implementar qualquer circuito combinacional pois eles são definidos a partir dessas operações.
A lógica sequencial é apresentada em \cite{harris} como uma construção hierárquica a partir do circuito SR Latch e lógica combinacional.
Como o próprio circuito SR é feito a partir de lógica combinacional, isso quer dizer que pode-se implementar os outros circuitos sequencias tais como D Latchs e Flip Flops, apenas usando as portas lógicas.

\subsubsection{Construindo processador SAP1}
A construção do processador SAP1 é feita a partir da construção de vários submodulos.
Como visto, o processador faz uso de um circuito subtrator e somador, registradores, memória rom, unidade de controle e contador.
Logo, para construir o processador deve-se construir esses subcircuitos e definir a classe processador como qualquer outro circuito.

A construção desse blocos a partir de portas lógicas está descrita em \cite{harris}.
Cada circuito foi construido da mesma maneira: crie uma classe herdando Base\_circuit; defina as entradas e saídas; implemente o método make seguindo as conexões descritas na literatura; usando testes unitários verifique que o circuito se comporta de acordo com a tabela verdade do mesmo.
O critério de aceitação para um circuito é que o circuito é aceito quando o valor de saída produzido pelo simulador para cada combinação de valores de entrada concorda com o valor esperado na teoria.
A lista de todos os subcircuitos feitos para codificar o processador é extensa.
O código fonte dos circuitos está disponível http://github.com/lodek/pdd.

Com todos os subcircuitos codificados, a construção do processador foi feita seguindo o diagrama de blocos do processador.
Foi criada uma bus central e todos os blocos foram conectados a ela.

Para executar programas no processador foi feito um script que: cria o objeto processador, carrega a memória ROM do processador com as instruções do programa, pulsa o clock pelo número de ciclos necessários para executar o programa e imprime o valor de saída do processador.
No total, foram escritos 2 programas que testam o funcionamento do processador.
Esses programas são apresentados nas tabelas \ref{tab-p1} e \ref{tab-p2}.

\begin{table}
  \caption{Programa 1 executado no processador virtual. Programa testa as instruções LDA e OUT.}
\begin{tabular}{ll}
  \hline
 Endereço inst. & Programa 1 \\
 \hline
 0x1 & 0x0F \#LDA end. 0x0F \\
 0x2 & 0xE0 \#OUT \\
 0x3 & 0x00 \\
 0x4 & 0x00  \\
 0x5 & 0x00  \\
 0x6 & 0x00  \\
 0x7 & 0x00  \\
 0x8 & 0x00  \\
 0x9 & 0x00  \\
 0xA & 0x00  \\
 0xB & 0x00  \\
 0xC & 0x00  \\
 0xD & 0x00  \\
 0xF & 0x03 \#valor LDA \\
 \hline
\end{tabular}
\label{tab-p1}
\end{table}


\begin{table}
  \caption{Programa 2 executado no processador virtual. Programa faz uso de todas as instruções disponiveis na arquitetura SAP1.}
\begin{tabular}{ll}
  \hline
 Endereço inst. & Programa 2 \\
 \hline
 0x1 & 0x0F \#LDA end. 0x0F \\
 0x2 & 0x1E \#ADD end. 0x0E \\
 0x3 & 0x2D \#SUB end. 0x0D \\
 0x4 & 0xE0 \#OUT \\
 0x5 & 0x00 \\
 0x6 & 0x00 \\
 0x7 & 0x00 \\
 0x8 & 0x00 \\
 0x9 & 0x00 \\
 0xA & 0x00 \\
 0xB & 0x00 \\
 0xC & 0x02 \#valor SUB \\
 0xD & 0x06 \#valor ADD \\
 0xF & 0x03 \#valor LDA \\
 \hline
\end{tabular}
\label{tab-p2}
\end{table}


O programa 1 na tabela \ref{tab-p1} testa a instrução LDA e a instrução OUT.
O programa carrega o registrador A com o valor no endereço 0x0F e em seguida executa a instrução OUT.
O comportamento esperado é que o valor no registrador O seja o mesmo valor contido no endereço 0x0F.

O programa 2 na tabela \ref{tab-p2} testa a instrução as instruções que restam, SUM e SUB.
Primeiramente o programa executa a instrução LDA, e carrega o registrador A com o valor $3$.
Em seguida o programa executa a instrução ADD e soma o valor contido no endereço 0x0E (0x06) ao valor em A.
Finalmente, o programa executa SUB e subtrai o valor em A com o valor no endereço 0x0D (0x02). 
O programa encerra com a instrução OUT.
Espera-se que o valor contido no registrador O seja 0x07 ($3+6-2$).

O processador foi considerado correto quando o valor obtido da execução de cada programa concordava com o valor esperado.


\subsection{ANÁLISE E DISCUSSÃO}
A ferramenta criada modela os circuitos digitais de uma maneira simplificada.
Circuitos são compostos de terminais de entrada e saída,  sub circuitos internos e uma rede de ligações.
Esse modelo implementado usando orientação a objeto resultou em uma descrição simplificada de um circuito.
Essa descrição faz com que a curva de aprendizado da ferramenta seja, teoricamente, pequena.

A utilização da hierarquia como parte do processo de construção dos circuitos resulta em circuitos modulares.
Isso é de grande vantagem pois facilita a interoperabilidade dos circuitos escritos por um ou mais usuários.
Assim, caso o usuário deseja construir um circuito complexo, a modularidade permite que ele separe a construção do circuito em módulos.
Trabalhar em módulos elimina repetição e permite que cada modulo seja testado isoladamente, fazendo uso de testes unitários.

Como foi visto, um requisito do simulador é que ele fosse capaz de simular um processador.
Usando a ferramenta, foi possível construir o processador e o mesmo produziu os resultados corretos. 
A modularidade facilitou muito no processo de construção pois cada modulo foi testado individualmente antes de serem unidos.
A construção do processador foi simples e resultou em um número moderado de linhas de código, foram menos de 40 linhas para conectar todos os módulos do processador.

O código do processador foi testado usando os programas descritos na seção anterior.
A execução correta desses programas demonstra que todas as instruções do processador funcionam adequadamente.
O snippet abaixo demonstra o resultado impresso no terminal para a simulação dos programas 1 e 2.
Observa-se que o valor de saída (out) para o programa 1 e 2 estão de acordo com o esperado, como foi descrito na metodologia.

\begin{lstlisting}
~/projects/pdd/builds/sap1 $ python run.py 
Tempo de execucao: 0:03:29.625620
resultado prog. 1 Processor: {'clk': '0x0', 'r': '0x0', 'out': '0x3', 'wout': '0x4'}
resultado prog. 2 Processor: {'clk': '0x0', 'r': '0x0', 'out': '0x7', 'wout': '0xe4'}
\end{lstlisting}

Um ponto negativo da metodologia textual usada é que para um circuito complexo como o processador, traçar as conexões é algo complexo.
A partir dessa experiência ficou claro que a carência de qualquer forma de representação gráfica do circuito pode ser prejudicial em algumas construções.
Uma possível solução seria adicionar um modulo que gera um diagrama de conexões a partir de um objeto circuito.

Embora exista uma dificuldade não esperada com a interface da ferramenta, existe um problema grave de performance que deve ser corrigido para viabilizar o seu uso.
Isso pode ser observado no tempo de execução dos 2 programas usados para testar o processador.
Cada processador executou menos de 5 instruções no total, a execução dos 2 programas escritos para o processador demorou um tempo inaceitável, 3.5 minutos, aproximadamente.
Essa quantia de tempo é astronômica dado a simplicidade das instruções realizadas.
Deve ser ressaltado que o processador SAP1 usa uma única bus, armazena apenas 16 palavras em memória, opera sobre palavras de 8 bit e possui uma ALU simples.
É passível de concluir que a simulação de um processador mais complexo, como um de 32 bits seria inviável.
Esse problema de performance é devido a metodologia hierárquica usada.

Como o elemento base de todos os circuito construídos no simulador foram as portas lógicas, isso significa que todas as operações do circuito utiliza as operações AND, OR e XOR do computador host.
Isso é um enorme desperdício dos diversos componentes de hardware já presentes no host.
Por questões de eficiência, o simulador poderia estar fazendo uso dos circuitos de operações matemática e memória embutidos no próprio computador.
Embora a construção desse circuitos a partir de portas lógicas seja um processo instrutivo, quando usados em um circuito complexo, isso prejudica a performance do simulador.
A correção desse problema é relativamente simples, basta reescrever as classes de circuitos somadores, subtratores, registradores e etc tal que esses circuitos usem os operadores presentes na linguagem Python.
Por exemplo, de maneira análoga ao que foi feito na classe AND, onde foi usando o operador \&, a classe do circuito somador usaria o operador "+".
Isso evita implementar essa operação usando portas lógicas.

A solução para o problema de performance indiretamente resolve uma outra inconveniência relacionada a inspeção de circuitos sequenciais.
A inspeção de um circuito combinacional pode ser feito de maneira conveniente na ferramenta, pois basta verificar o seus valores de entrada para identificar a saída.
Essa inspeção é feita usando o comando print, que apresentada os sinais de todas os terminais de entrada e saída.
Porém devido a natureza da lógica sequencial, seria interessante manter uma lista dos valores anteriores de um circuito sequencial, sendo assim possível fazer uma análise temporal de um circuito de maneira simplificada.
Por exemplo, suponha a existência de um circuito contador configurável, seria conveniente poder alterar e armazenar um sinal no contador sem alterar o valor na bus do circuito.
Isso permite que seja possível induzir um estado em um circuito sequencial, o que facilita o processo de debug.
Para implementar isso, basta adicionar um método de configuração nos circuitos sequenciais, tal como um circuito FlipFlop. 

Em conclusão, a ferramenta construída provou-se capaz de realizar a simulação de diversos circuitos digitais.
Na construção dela foi utilizado um paradigma orientado a objetos para descrever um circuito, pois acreditava-se que esse paradigma iria simplificar a modelagem de circuitos.
O escopo do projeto foi delimitado para a construção da ferramenta, como consequência não se sabe se esta ferramenta é de fato mais simples para um novato na disciplina.
Um interessante próximo passo seria disponibilizar esta ferramenta em um ambiente web a fim de realizar um teste alfa e receber feedback.
Seria possível desenvolver um webapp com um interpretador Python que contém a biblioteca instalada no ambiente virtualizado.
Essa metodologia de uso seria ideal pois não requer a instalação da biblioteca ou da linguagem Python no computador do usuário.



\newpage
\section{CRONOGRAMA REALIZADO}
\section{CRONOGRAMA REALIZADO}
\section{CRONOGRAMA REALIZADO}
\input{tables/cronograma}



\newpage
\input{tex/consideracoes}
\newpage
\addcontentsline{toc}{section}{6   PARECER DO SUPERVISOR TÉCNICO}
\parecer

\newpage
\printbibliography
\end{document}
