\begin{table}
  \caption{Programa 1 executado no processador virtual. Programa testa as instruções LDA e OUT.}
\begin{tabular}{ll}
  \hline
 Endereço inst. & Programa 1 \\
 \hline
 0x1 & 0x0F \#LDA end. 0x0F \\
 0x2 & 0xE0 \#OUT \\
 0x3 & 0x00 \\
 0x4 & 0x00  \\
 0x5 & 0x00  \\
 0x6 & 0x00  \\
 0x7 & 0x00  \\
 0x8 & 0x00  \\
 0x9 & 0x00  \\
 0xA & 0x00  \\
 0xB & 0x00  \\
 0xC & 0x00  \\
 0xD & 0x00  \\
 0xF & 0x03 \#valor LDA \\
 \hline
\end{tabular}
\label{tab-p1}
\end{table}


\begin{table}
  \caption{Programa 2 executado no processador virtual. Programa faz uso de todas as instruções disponiveis na arquitetura SAP1.}
\begin{tabular}{ll}
  \hline
 Endereço inst. & Programa 2 \\
 \hline
 0x1 & 0x0F \#LDA end. 0x0F \\
 0x2 & 0x1E \#ADD end. 0x0E \\
 0x3 & 0x2D \#SUB end. 0x0D \\
 0x4 & 0xE0 \#OUT \\
 0x5 & 0x00 \\
 0x6 & 0x00 \\
 0x7 & 0x00 \\
 0x8 & 0x00 \\
 0x9 & 0x00 \\
 0xA & 0x00 \\
 0xB & 0x00 \\
 0xC & 0x02 \#valor SUB \\
 0xD & 0x06 \#valor ADD \\
 0xF & 0x03 \#valor LDA \\
 \hline
\end{tabular}
\label{tab-p2}
\end{table}
