\subsection{ANÁLISE E DISCUSSÃO}
A ferramenta criada modela os circuitos digitais de uma maneira simplificada.
Circuitos são compostos de terminais de entrada e saída,  sub circuitos internos e uma rede de ligações.
Esse modelo implementado usando orientação a objeto resultou em uma descrição simplificada de um circuito.
Essa descrição faz com que a curva de aprendizado da ferramenta seja, teoricamente, pequena.

A utilização da hierarquia como parte do processo de construção dos circuitos resulta em circuitos modulares.
Isso é de grande vantagem pois facilita a interoperabilidade dos circuitos escritos por um ou mais usuários.
Assim, caso o usuário deseja construir um circuito complexo, a modularidade permite que ele separe a construção do circuito em módulos.
Trabalhar em módulos elimina repetição e permite que cada modulo seja testado isoladamente, fazendo uso de testes unitários.

Como foi visto, um requisito do simulador é que ele fosse capaz de simular um processador.
Usando a ferramenta, foi possível construir o processador e o mesmo produziu os resultados corretos. 
A modularidade facilitou muito no processo de construção pois cada modulo foi testado individualmente antes de serem unidos.
A construção do processador foi simples e resultou em um número moderado de linhas de código, foram menos de 40 linhas para conectar todos os módulos do processador.

O código do processador foi testado usando os programas descritos na seção anterior.
A execução correta desses programas demonstra que todas as instruções do processador funcionam adequadamente.
O snippet abaixo demonstra o resultado impresso no terminal para a simulação dos programas 1 e 2.
Observa-se que o valor de saída (out) para o programa 1 e 2 estão de acordo com o esperado, como foi descrito na metodologia.

\begin{lstlisting}
~/projects/pdd/builds/sap1 $ python run.py 
Tempo de execucao: 0:03:29.625620
resultado prog. 1 Processor: {'clk': '0x0', 'r': '0x0', 'out': '0x3', 'wout': '0x4'}
resultado prog. 2 Processor: {'clk': '0x0', 'r': '0x0', 'out': '0x7', 'wout': '0xe4'}
\end{lstlisting}

Um ponto negativo da metodologia textual usada é que para um circuito complexo como o processador, traçar as conexões é algo complexo.
A partir dessa experiência ficou claro que a carência de qualquer forma de representação gráfica do circuito pode ser prejudicial em algumas construções.
Uma possível solução seria adicionar um modulo que gera um diagrama de conexões a partir de um objeto circuito.

Embora exista uma dificuldade não esperada com a interface da ferramenta, existe um problema grave de performance que deve ser corrigido para viabilizar o seu uso.
Isso pode ser observado no tempo de execução dos 2 programas usados para testar o processador.
Cada processador executou menos de 5 instruções no total, a execução dos 2 programas escritos para o processador demorou um tempo inaceitável, 3.5 minutos, aproximadamente.
Essa quantia de tempo é astronômica dado a simplicidade das instruções realizadas.
Deve ser ressaltado que o processador SAP1 usa uma única bus, armazena apenas 16 palavras em memória, opera sobre palavras de 8 bit e possui uma ALU simples.
É passível de concluir que a simulação de um processador mais complexo, como um de 32 bits seria inviável.
Esse problema de performance é devido a metodologia hierárquica usada.

Como o elemento base de todos os circuito construídos no simulador foram as portas lógicas, isso significa que todas as operações do circuito utiliza as operações AND, OR e XOR do computador host.
Isso é um enorme desperdício dos diversos componentes de hardware já presentes no host.
Por questões de eficiência, o simulador poderia estar fazendo uso dos circuitos de operações matemática e memória embutidos no próprio computador.
Embora a construção desse circuitos a partir de portas lógicas seja um processo instrutivo, quando usados em um circuito complexo, isso prejudica a performance do simulador.
A correção desse problema é relativamente simples, basta reescrever as classes de circuitos somadores, subtratores, registradores e etc tal que esses circuitos usem os operadores presentes na linguagem Python.
Por exemplo, de maneira análoga ao que foi feito na classe AND, onde foi usando o operador \&, a classe do circuito somador usaria o operador "+".
Isso evita implementar essa operação usando portas lógicas.

A solução para o problema de performance indiretamente resolve uma outra inconveniência relacionada a inspeção de circuitos sequenciais.
A inspeção de um circuito combinacional pode ser feito de maneira conveniente na ferramenta, pois basta verificar o seus valores de entrada para identificar a saída.
Essa inspeção é feita usando o comando print, que apresentada os sinais de todas os terminais de entrada e saída.
Porém devido a natureza da lógica sequencial, seria interessante manter uma lista dos valores anteriores de um circuito sequencial, sendo assim possível fazer uma análise temporal de um circuito de maneira simplificada.
Por exemplo, suponha a existência de um circuito contador configurável, seria conveniente poder alterar e armazenar um sinal no contador sem alterar o valor na bus do circuito.
Isso permite que seja possível induzir um estado em um circuito sequencial, o que facilita o processo de debug.
Para implementar isso, basta adicionar um método de configuração nos circuitos sequenciais, tal como um circuito FlipFlop. 

Em conclusão, a ferramenta construída provou-se capaz de realizar a simulação de diversos circuitos digitais.
Na construção dela foi utilizado um paradigma orientado a objetos para descrever um circuito, pois acreditava-se que esse paradigma iria simplificar a modelagem de circuitos.
O escopo do projeto foi delimitado para a construção da ferramenta, como consequência não se sabe se esta ferramenta é de fato mais simples para um novato na disciplina.
Um interessante próximo passo seria disponibilizar esta ferramenta em um ambiente web a fim de realizar um teste alfa e receber feedback.
Seria possível desenvolver um webapp com um interpretador Python que contém a biblioteca instalada no ambiente virtualizado.
Essa metodologia de uso seria ideal pois não requer a instalação da biblioteca ou da linguagem Python no computador do usuário.
