\section{CONSIDERAÇÕES FINAIS}
Durante o projeto de estágio foi construído uma ferramenta em Python para simulação de circuitos digitais.
Essa ferramenta faz uso do paradigma de programação orientada a objetos para descrever um circuito.
Sendo assim, a implementação de um circuito é feita a partir de classes codificadas pelos usuários.
Essa metodologia garante circuitos modulares e regulares pois faz uso do conceito de herança para garantir uma interface consistente entre circuitos.

A ferramenta desenvolvida foi utilizada para construir um processador SAP1, uma arquitetura didática.
A ferramenta se mostrou capaz de simular o processador corretamente, porém foi notável um problema de performance.
Esse problema está relacionado a metodologia hierárquica empregada pela ferramenta.
Possíveis soluções para o problema foram abordadas na seção de analise.

O intuito da ferramenta é criar uma descrição simplificada de circuitos para que estudantes possam construir circuitos de maneira simples com o objetivo de facilitar ao processo de aprendizado.
Espera-se que um estudante possa construir um processador, a partir de uma bibliografia, sem que isso seja uma tarefa complexa e longa.
Sendo assim, essa ferramenta tenta substituir, até certo ponto, a construção de um processador em uma breadboard (processo longo) ou em uma linguagem HDL (processo complexo).

\begin{flushright}
\vspace{3cm}
Curitiba\\
\today\\
\vspace{3cm}
\noindent \rule{0.5\textwidth}{0.4pt}\\
Bruno Gomes\\
\end{flushright}
