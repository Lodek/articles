\subsection{INTRODUÇÃO}
\subsubsection{Problema de pesquisa}
O projeto de pesquisa engloba a construção de um simulador de circuitos digitais em Python.
O simulador deve usar objetos para modelar circuitos digitais.
A representação de um circuito é feita por classes escritas pelo usuário.
Um requisito é que o simulador deve ser capaz de simular um processador.
Para atender a esse requisito foi construído um processador virtual que implementa a arquitetura SAP1.

\subsubsection{Objetivos}
a) Objetivo Geral:\\
Construir uma ferramenta para simulação de circuitos digitais e simular um processador na ferramenta criada.\\
b) Objetivos Específicos:
\begin{itemize}
\item Definir características do simulador e delimitar escopo do projeto.
\item Modelar circuitos digitais como objetos.
\item Codificar simulador.
\item Construir processador.
\end{itemize}

\subsubsection{Justificativa}

A disciplina de circuitos digitais é a base que explica o funcionamento dos sistemas microprocessados atuais.
Construir um circuito digital é aplicar os conhecimentos teóricos de maneira prática, um ato fundamental para o aprendizado.
Para construir circuitos digitais existem três maneiras bem estabelecidas: construir o circuito físico utilizando chipes fabricados, construir o circuito em uma ferramenta de simulação gráfica e construir o circuito em uma ferramenta HDL.
Será explorado as vantagens e desvantagens de cada método e em seguida será proposto uma alternativa que utiliza a orientação a objetos para construir circuitos.

Os circuitos físicos são normalmente construídos em protoboards utilizando chipes e jumpers.
Cada chipe é um circuito digital encapsulado que realiza uma operação imutável.
Existem um arsenal de chipes que realizam operações diferentes.
Embora seja o método clássico, a medida que os circuitos crescem em complexidade, as chaces de cometer um erro aumentam devido ao número de conexões que devem ser feitas.
Outra desvantagem é o tempo necessário para construir um circuito físico.
O processo não é eficiente, demandando muito tempo para verificar conexões e procurar por informações em datasheets.
Outro problema refere-se à dificuldade de abstrair e modularizar um circuito, caso ele não exista pré fabricado será obrigação do usuário de construir esse circuito sempre que necessário, o que é ineficiente.
A vantagem desse método é que ele é usado na fase de prototipação de dispositivos eletrônicos.
Devido a essas características muitas vezes são adotadas ferramentas de simulação ao invés de circuitos físicos.

Existem uma enorme quantidade de ferramentas para simulação porém nem todas são apropriadas para todos os públicos.
No geral, as ferramentas de simulação mais simples fazem uso de diagramas de conexão.
Nesses diagramas, o usuário deve posicionar os blocos correspondentes aos circuitos e traçar as conexões entre eles.
Essa metodologia é pouco melhor do que os circuitos físicos, sendo repetitiva e possuindo baixa abstração.
Embora existam inúmeras ferramentas de simulação gráficas, o foco principal será nas HDL (linguagem de descrição de hardware).

As HDL são similares a linguagens de programação e são usadas para descrever circuitos.
A descrição de circuitos em HDL é feita por arquivos de texto, análogos a programas de computador.
Uma vez escritos, os circuitos podem ser usados para simulação ou síntese, sendo uma solução completa para o processo de design.
Segundo \textcite{harris}, as HDLs foram criadas na década de 90 para abstrair o processo de design dos engenheiros responsáveis pelos projetos de circuitos digitais, pois elas realizam o processo de otimização e simplificação do circuito.
Devido a isso, as HDLs agilizam o processo de design e poupam tempo do engenheiro.
O inconveniente dessas ferramentas é que elas foram projetadas para suprir a necessidade dos engenheiros projetando circuitos digitais para o mercado e, não para a simulação de circuitos para fins didáticos.
A consequência disso é que as HDL possuem mais funcionalidades que necessário para um contato inicial e por terem sido projetadas para a indústria, a representação de circuitos não é necessariamente intuitiva.

Deseja-se criar uma ferramenta que possui características similares às HDL, porém que apliquem um modelo mais simples e consequentemente mais limitado.
Essa nova ferramenta seria, idealmente, simples, a fim de ser usada por iniciantes em circuitos digitais.
Ela seria uma alternativa para estudantes de circuitos digitais que desejam construir um processador sem ter que batalhar com a ferramenta ou com o ambiente de desenvolvimento.

Após a construção da ferramenta, ela será usada para simular um processador que implementa a arquitetura SAP1.
A ferramenta deve ser capaz de cumprir essa tarefa pois, segundo \textcite{harris} a construção de um processador é quase um rito de passagem para qualquer engenheiro da computação.
Então, deseja-se que essa ferramenta seja capaz de suprir essa necessidade quase que universal dos estudantes de engenharia.
