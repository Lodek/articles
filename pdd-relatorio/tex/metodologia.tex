\subsection{METODOLOGIA}
\subsubsection{Desenvolvimento PDD}
\paragraph{Representação}
A representação de um circuito é de extrema importância para o simulador.
Uma representação complexa e detalhada torna a ferramenta menos acessível a novos usuários.
Sendo assim, a representação escolhida tentou ser simples e familiar, pois modela circuitos de maneira análoga ao apresentado nas literaturas.

Definir a representação foi o primeiro passo do processo criativo pois ela define a interface do código.
A interface definida passou por uma série de iterações até chegar em um resultado que o autor considerou agradável e compreensível.
Como um dos objetivos é desenvolver uma ferramenta amigável, primeiramente será demonstrado snippets de código do modelo implementado.

O código abaixo cria uma porta do tipo AND e altera os sinais de entrada dessa porta.
Percebe-se que a porta é criada usando a notação usual de construção de objetos usada nas linguagens orientada a objetos.
A porta AND possui duas entradas (A e B) e uma saída (Y), e como foi visto anteriormente cada uma dessas conexões é um terminal.
A interação com os terminais é feita a partir do operador ponto (usado para acessar atributos).
No código acessamos a bus ligada ao terminal "A" da porta lógica e em seguida modificamos o valor do sinal dela.
Essa operação foi executada de maneira enxuta e legível, sem dificuldades para o usuário.
O mesmo procedimento é executado na porta "B" e finalmente executa-se a função print para verificar o sinal no terminal "Y".
Também é possível chamar print na porta como um todo e é impresso todas as informações referentes ao circuito.

\begin{lstlisting}
  porta = AND() #cria porta do tipo AND e armazena a referencia na variavel porta
  porta.a.signal = 1 #modifica o sinal presente na Bus conectada ao terminal A
  porta.b = 1
  print(porta.y) #imprime informacao do terminal Y
  print(porta) #impreme informacao do circuito
\end{lstlisting}

A composição de circuitos também é feita a partir de operações nos objetos.
O código abaixo cria duas portas AND, p1 e p2, e conecta a saída de p1 à entrada de p2.
A conexão dos circuitos é feita a partir do método "connect".
Esse método faz uso de argumentos do tipo chave-valor.
O método recebe um conjunto chave-valor onde a chave é o nome de um terminal do circuito e o valor é a bus que será ligada ao terminal.
A conexão entre os dois circuitos foi feita em uma linha de código de maneira clara e simples.

\begin{lstlisting}
  p1, p2 = AND(), AND()
  p2.connect(a=p1.y) #Liga saida de p1 na entrada 'a' de p2
\end{lstlisting}

O próximo snippet utiliza os conceitos apresentados até agora e aplica modularidade ao processo.
O código a seguir define um novo circuito, ou seja uma classe, e a ele damos o nome "NovoCircuito".
NovoCircuito herda de uma classe abstrata disponibilizada pela ferramenta e implementa o mesmo circuito do exemplo anterior (duas portas AND conectadas).
A vantagem de criar uma classe para esse circuito é que ele fica reutilizável e basta criar uma nova instância do objeto, além de aplicar o conceito de hierarquia, ou seja, construímos um circuito mais complexo a partir de circuitos simples.
A classe herdeira deve definir os nomes dos terminais de entrada e saída, e implementar o método "make".
Os atributos "input\_labels" e "output\_labels" devem ser uma lista de strings que definem os nomes dos terminais de entrada e saída, respectivamente.
O método make é aonde as conexões entre sub circuitos são definidas.
Algo de grande importância no método make é que o usuário deve chamar o método \texttt{self.get\_inputs()}.
Esse método retorna um struct contendo as buses internas ao circuito sendo construído.
Esse passo é essencial, pois é a partir dessas buses que é feita a conexão do NovoCircuito com os subcircuitos presentes nele.
Sem os objetos retornados pelo \texttt{get\_inputs}, NovoCircuito seria uma caixa preta com entradas ligadas a nenhum subcircuito.

\begin{lstlisting}
class NovoCircuito(BaseCircuit):
    input_labels = ['a', 'b', 'c'] #lista de nomes dos terminais de entrada
    output_labels = ['y'] #lista de nomes dos terminais de saida

    def make(self):
        i = self.get_inputs() #objeto que contem as buses internas ao circuito NovoCircuito.
        and_1 = AND()
        and_1.connect(a=i.a, b=i.b) #Conecta as buses de entrada a e b ao circuito and_1
        and_2 = AND(a=and_1.y, b=i.c) 
        self.set_outputs(y=and_2.y) #Conecta a saida do circuito and_2 a saida 'Y' do NovoCircuito
\end{lstlisting}

Finalmente, o último snippet demonstra a utilização de NovoCircuito.
O processo é idêntico à criação de uma porta AND.

\begin{lstlisting}
  novo = NovoCircuito()
  novo.a.signal = 1
  novo.c.signal = 0
  print(novo)
\end{lstlisting}

Os quatro snippets acima são curtos porém eles, de certa forma, são tudo o que um usuário precisa entender sobre a ferramenta.
Os snippets introduzem a "linguagem" da ferramenta, com essa linguagem o usuário pode modelar qualquer circuito que ele queira.
A simples representação escolhida é o diferencial da ferramenta e a partir dela, pode-se criar circuitos de complexidade incremental, usando modularidade e hierarquia, e tudo isso com uma interface de código consistente.

\paragraph{Arquitetura do simulador}
As classes usadas para definir o modelo foram introduzidas no referêncial teórico.
Nessa seção serão apresentadas as interfaces e as funcionalidades dessas classes.

A classe Wire, modela um fio que armazena um bit.
Wire armazena esse bit em uma variável booleana que pode ser modificado usando os metodos \texttt{set()} e \texttt{reset()}.
Objetos do tipo Wire são imutáveis e essa característica vai ser fundamental para a implementação da lógica de atualização do simulador.

A classe Bus representa um conjunto de fios (objetos do tipo Wire).
Essa classe define uma abstração em cima da classe Wire, fazendo com que Wire não seja relevante ao usuário.
Ao invés de armazenar 1 bit de informação, Bus armazena n bits, onde n é o número de fios que constituem a mesma.
Sendo assim, Bus possui o atributo sinal que representa um número construído a partir dos bits armazenados nos fios.
Bus possui o atributo wires que é uma lista de objetos wire.
Como a bus representa uma informação multibit a ordenação dos fios na lista é importante.
O fio na posição $n$ na lista representa o bit na posição $n$.
Objetos do tipo Bus são mutáveis, um usuário pode realizar operações em Bus para construir outras Bus a partir da mesma.
Por exemplo, suponha que uma Bus W tenha 8 wires, o usuário pode construir uma Bus Y que tenha os 4 primeiros wires de W.
Isso foi feito para facilitar o acesso a bits em uma Bus, sendo esse um caso de uso comum na construção de circuitos digitais.
Finalmente, para alterar o valor de um sinal na Bus basta usar o comando bus.signal = 3.

A classe Terminal é usada para definir entradas e saídas em um circuito.
O objeto Terminal pode ser visualizado como uma porta lógica do tipo Buffer onde existe uma entrada e uma saída, porém Terminal possui algumas conveniências.
O terminal pode funcionar como um inversor e um buffer de três estados, basta configurá-lo corretamente na fase de construção do objeto.
Tendo um único objeto que apresente todas essas funcionalidades é desejável pois fica extremamente fácil de reutilizar circuitos.
Suponha que tenha sido construído um circuito multiplexador comum e que em um momento futuro o usuário precise inverter uma das entradas do multiplexador ou adicionar um buffer de três estados na saída.
Assim, as conveniências da classe Terminal evita a necessidade de construir uma outra classe ou adicionar uma porta NOT no circuito, basta informar os parâmetros corretos na hora de inicializar o objeto.
Os atributos de Terminal são "bus-in" e "bus-out", que definem as buses de entrada e saída, respectivamente, "en" que define a configuração do buffer de três estados, "invert" que configura o terminal como uma porta NOT e "size" que define o tamanho das buses que podem ser conectados ao terminal.
O método propagate replica o sinal da bus de entrada para a bus de saída, dependendo da configuração do terminal.
Por exemplo, caso o comportamento de três estados do terminal esteja com impedância alta na saída, o método propagate não mudará o valor da bus de saída;
ou caso o terminal esteja configurado para negar a saída, o sinal de saída será o completo do sinal de entrada.

Finalmente, temos a classe que modela circuitos, a classe BaseCircuit.
A classe BaseCircuit define uma classe abstrata.
Essa classe implementa um construtor que, a partir dos atributos "input\_labels" e "output\_labels", automaticamente cria terminais de entrada e saída, buses para esses terminais e chama o método \texttt{make()}, responsável pela criação e configuração dos subcircuitos.
Fora o método make, essa classe implementa o método update, que é responsável por atualizar o circuito, e nesse ponto existe uma sutileza do design da ferramenta.

Deve-se lembrar que o simulador faz uso da  hierarquia, ou seja, circuitos são construídos a partir da combinação de circuitos mais simples.
Isso faz com que a atualização do circuito seja, na verdade, a atualização dos subcircuitos que o compõe. 
Como os circuitos são construídos a partir das portas lógicas, isso faz com que todo o trabalho de atualização seja feito a um nível de AND, OR e XOR.
Sendo assim, para implementar o método de atualização da classe BaseCircuito, o circuito deve chamar o método propagate para todos os terminais de entrada e chamar o método de atualização nos subcircuitos.
Esse ciclo de transferência de responsabilidades se repete até chegar nas portas lógicas, onde são realizados as operações da álgebra booleana sobre os sinais de entrada.

Agora que foram explicadas todas as relações entre as classes do modelo, o único passo que falta é a implementação das portas lógicas.
Como toda a complexidade de criar uma interface constante esta definido na classe BaseCircuit, a implementação das portas lógicas é feita de maneira similar aos exemplos dados na seção anterior.
A única diferençã é que as portas lógicas dão override no método update, pois como visto acima, elas têm uma responsabilidade diferente aos outros circuitos.

O snippet abaixo contém o código usado para implementar a porta AND.
Nota-se que o código faz uso de herança e aproveita os métodos ja implementados na classe pai para atualizar as buses internas do circuito.
A operação lógica AND é feita usando o operador "\&" presente por padrão na linguagem Python.
As portas lógicas OR e XOR foram construidas de maneira análoga, substituindo o operador \& pelos operadores $|$ e \^{}, para a porta OR e XOR, respectivamente

\begin{lstlisting}
class AND(BaseCircuit):
    """
    Implementacao da porta AND.
    Entradas: a e b
    Saidas: y
    Classe realiza override do metodo update e realiza operacoes binarias sobre os sinais das buses
    """
    input_labels = 'a b'.split()
    output_labels = 'y'.split()
    def update(self):
        super().update() #chama o metodo update na classe pai, propaga o sinais de entrada para as buses internas
        i = self.get_inputs() #recebe buses internas
        output_signal = i.a.signal & i.b.signal #realiza a operacao binaria and sobre os sinais de entrada
        i.y.signal = output_signal #altera o sinal da bus de saida
        super().update() #novamente chama o metodo update para atualizar o sinal da bus de saida
\end{lstlisting}

Foi feita uma descrição alto nível das classes, sem entrar em particularidades de implementação na linguagem Python.
As classes como foram apresentadas acima definem o comportamento do modelo do simulador.
O código fonte completo de todas as classes, como foi explicado acima está disponível em http://github.com/lodek/pdd.

\paragraph{Back end}
O único elemento que resta para o funcionamento do simulador é uma entidade que automatize o processo de atualização de circuitos.
Em um circuito eletrônico digital real, quando o valor entrada de um circuito muda, após o atraso propagacional do mesmo, o valor de saída muda.
Esse comportamento foi usado como inspiração para solucionar o problema de atualização do processador e a solução foi implementada o padrão Observer.

Como foi visto, o Observer é usado quando existe uma dependência entre objetos.
O uso previsto do Observer é de uma relação um para muitos, porém o caso de uso no simulador é uma relação de muitos para muitos.
Para solucionar o problema acima, foi definido que um circuito depende de um conjunto de fios (objetos Wire).
O padrão Observer então permite que cada vez que um Wire seja modificado, esse informe aos seus dependentes e os atualize de maneira automática.
Nesse caso, foi definido que ao alterar o valor de entrada de qualquer circuito, ou seja, os valores dos objetos Wire, isso gera um evento.

A implementação do padrão Observer foi feita a partir das classes "Updater" e "Event".
A classe Event é um objeto simples que contém apenas o atributo source, uma referência para o Wire que criou o evento.
A classe Updater possui os atributos relations e events, e possui o método update.
O atributo relations é uma hashtable onde a chave é um objeto do tipo Wire e o valor é uma lista de objetos do tipo Circuit que são os dependentes do objeto Wire.
Já o atributo events é uma lista do tipo FIFO de objeto Event.
O método update inicia o ciclo de atualização do simulador, uma versão simplificada do algorítimo de atualização é apresentada a seguir.
\begin{lstlisting}
while updater.events: #iterar sobre fifo de eventos
    eventos = updater.events.pop() #remover primeiro evento
    dependentes = updater.relations[event.wire] #lista de circuitos dependentes do objeto wire
    for dependente in dependentes: #atualizar dependentes
        dependente.update()
\end{lstlisting}
O algorítimo de atualização foi escrito em 6 linhas.
Essa implementação simples é graças ao uso de eventos e do padrão Observer que separa a responsabilidade do processo de atualização entre as classes.

Aplicando o algoritmo acima para o circuito multiplexador (fig. \ref{f-hmux}), suponha que foi modificado o sinal de D0.
Por construção, A0 notifica o Updater com um evento.
Em seguida, é iniciado o processo de atualização usando updater.update().
O Observer irá identificar o evento gerado por D0 e então verifica os dependentes de D0.
Como M1 é dependente de D0, o Observer atualiza M1, que por sua vez modifica Y1, que por sua vez notifica o Observer.
Novamente, o Observer identifica um evento presente no FIFO e verifica os dependentes de Y1.
Esse processo se repete até que o FIFO de eventos esteja vazio.
Percebe-se que a alteração de uma Bus pode iniciar um efeito cascata, fazendo com que uma série de circuitos sejam atualizados.
O uso do Observer permite que esse processo de atualização seja feito de maneira simples.

Alguns pontos são importantes a respeito desta implementação.
O cadastro dos fios no Observer é feito pela classe BaseCircuit durante a construção do objeto, tal que o usuário não precisa conhecer o backend.
Também é importante ressaltar que o Observer usa objetos do tipo Wire, e não Bus, como fontes de eventos pois os objetos Bus são mutáveis, o que impossibilita o armazenamento das relações entre circuitos.

\paragraph{Front end}
Foram definidas duas maneiras de interação com o simulador.
Uma das maneiras é um ambiente tipo sandbox em que o usuário têm uma interação dinâmica com a ferramenta usando um REPL (Read-Evaluate-Print-Loop).
Como Python é uma linguagem interpretada, a própria linguagem oferece o REPL para ser usado.
A outra interface prevista é que a simulação dos circuitos será feita a partir de scripts escritos pelo usuário.
Essa metodologia é um pouco menos intuitiva porém é muito mais eficiente, já que uma vez criado o script, basta executá-lo novamente para recriar a simulação.

Para fazer uso do ambiente sandbox, o computador deve estar com Python instalado e a biblioteca escrita.
Sendo atendido esses requisitos de ambiente, basta que o usuário inicie o interpretador Python, importe a biblioteca e chame o método de configuração inicial.
Sendo feito essa configuração inicial o usuário pode criar e interagir com circuitos como foi descrito na seção de interface.
O snippet abaixo mostra os imports que devem ser feitos e o simulador funcionando para uma porta AND.

\begin{lstlisting}
Python 3.7.4 (default, Jul 16 2019, 07:12:58) 
Type 'copyright', 'credits' or 'license' for more information
IPython 7.6.1 -- An enhanced Interactive Python. Type '?' for help.

In [1]: import pdd

In [2]: from pdd.combinational_blocks import *

In [4]: pdd.core.Wire.auto_update = True

In [24]: a = AND()

In [25]: a
Out[25]: Gate AND: a=0x0; b=0x0; y=0x0; 

In [26]: a.a.set()

In [27]: a
Out[27]: Gate AND: a=0x1; b=0x0; y=0x0; 

In [28]: a.b.set()

In [29]: a
Out[29]: Gate AND: a=0x1; b=0x1; y=0x1; 
\end{lstlisting}

A outra possibilidade de uso é através de scripts.
O uso de scripts garante que nada seja perdido, todas as instruções sempre estarão salvas nele.
Por exemplo, um usuário pode definir um módulo Python com circuitos escritos por ele e uma função que realize a manipulação dos sinais de maneira automatizada.
Isso permite que, em caso de erros na implementação do circuito, basta abrir o script, corrigi-lo e executá-lo novamente.
Um uso mais avançado, porém possível, é usar uma metodologia tipo TDD (Test Driven Development) para desenvolver um circuito.

Como exemplo de script que pode ser escrito, o snippet abaixo é um script que implementa um circuito e apresenta os valores impressos no terminal.
O script implementa um circuito multiplexador de 2 entrada como é apresentado em \cite{harris}.
Nota-se que o circuito "d0\_and" faz uso da funcionalidade inversora da classe terminal para evitar o uso de uma porta NOT.
Apos a definição do circuito, cria-se um objeto do tipo SimpleMux e modifica-se os sinais de entrada do circuito.
O snippet contém também os valores de saída apresentados no terminal.

\begin{lstlisting}
import pdd
from pdd.combinational_blocks import *
from pdd.dl import BaseCircuit
pdd.core.Wire.auto_update = True

class SimpleMux(BaseCircuit):
    """
    Circuito multiplexer com 2 entrada e 1 linha de select
    """
    input_labels = 'd0 d1 s'.split()
    output_labels = 'y'.split()

    def make(self):
        i = self.get_inputs()
        d0_and = AND(a=i.d0, b=i.s, bubbles=['b']) #AND com entrada B invertida
        d1_and = AND(a=i.d1, b=i.s)
        select_or = OR(a=d0_and.y, b=d1_and.y)
        self.set_outputs(y=select_or.y)

mux = SimpleMux()
print(mux)
mux.d0 = 1
print(mux)
mux.s = 1
print(mux)

#######################################
#Output do terminal
#######################################

~/projects/pdd $ python mux_script.py
SimpleMux: {'d0': '0x0', 'd1': '0x0', 's': '0x0', 'y': '0x0'}
SimpleMux: {'d0': '0x1', 'd1': '0x0', 's': '0x0', 'y': '0x1'}
SimpleMux: {'d0': '0x1', 'd1': '0x0', 's': '0x1', 'y': '0x0'}
\end{lstlisting}

Isso encerra a definição e construção do simulador.
Como visto acima, foi implementado um circuito multiplexador construído a partir de portas lógicas.
Partindo das portas AND, OR e XOR é possível implementar qualquer circuito combinacional pois eles são definidos a partir dessas operações.
A lógica sequencial é apresentada em \cite{harris} como uma construção hierárquica a partir do circuito SR Latch e lógica combinacional.
Como o próprio circuito SR é feito a partir de lógica combinacional, isso quer dizer que pode-se implementar os outros circuitos sequencias tais como D Latchs e Flip Flops, apenas usando as portas lógicas.

\subsubsection{Construindo processador SAP1}
A construção do processador SAP1 é feita a partir da construção de vários submodulos.
Como visto, o processador faz uso de um circuito subtrator e somador, registradores, memória rom, unidade de controle e contador.
Logo, para construir o processador deve-se construir esses subcircuitos e definir a classe processador como qualquer outro circuito.

A construção desse blocos a partir de portas lógicas está descrita em \cite{harris}.
Cada circuito foi construido da mesma maneira: crie uma classe herdando Base\_circuit; defina as entradas e saídas; implemente o método make seguindo as conexões descritas na literatura; usando testes unitários verifique que o circuito se comporta de acordo com a tabela verdade do mesmo.
O critério de aceitação para um circuito é que o circuito é aceito quando o valor de saída produzido pelo simulador para cada combinação de valores de entrada concorda com o valor esperado na teoria.
A lista de todos os subcircuitos feitos para codificar o processador é extensa.
O código fonte dos circuitos está disponível http://github.com/lodek/pdd.

Com todos os subcircuitos codificados, a construção do processador foi feita seguindo o diagrama de blocos do processador.
Foi criada uma bus central e todos os blocos foram conectados a ela.

Para executar programas no processador foi feito um script que: cria o objeto processador, carrega a memória ROM do processador com as instruções do programa, pulsa o clock pelo número de ciclos necessários para executar o programa e imprime o valor de saída do processador.
No total, foram escritos 2 programas que testam o funcionamento do processador.
Esses programas são apresentados nas tabelas \ref{tab-p1} e \ref{tab-p2}.

\begin{table}
  \caption{Programa 1 executado no processador virtual. Programa testa as instruções LDA e OUT.}
\begin{tabular}{ll}
  \hline
 Endereço inst. & Programa 1 \\
 \hline
 0x1 & 0x0F \#LDA end. 0x0F \\
 0x2 & 0xE0 \#OUT \\
 0x3 & 0x00 \\
 0x4 & 0x00  \\
 0x5 & 0x00  \\
 0x6 & 0x00  \\
 0x7 & 0x00  \\
 0x8 & 0x00  \\
 0x9 & 0x00  \\
 0xA & 0x00  \\
 0xB & 0x00  \\
 0xC & 0x00  \\
 0xD & 0x00  \\
 0xF & 0x03 \#valor LDA \\
 \hline
\end{tabular}
\label{tab-p1}
\end{table}


\begin{table}
  \caption{Programa 2 executado no processador virtual. Programa faz uso de todas as instruções disponiveis na arquitetura SAP1.}
\begin{tabular}{ll}
  \hline
 Endereço inst. & Programa 2 \\
 \hline
 0x1 & 0x0F \#LDA end. 0x0F \\
 0x2 & 0x1E \#ADD end. 0x0E \\
 0x3 & 0x2D \#SUB end. 0x0D \\
 0x4 & 0xE0 \#OUT \\
 0x5 & 0x00 \\
 0x6 & 0x00 \\
 0x7 & 0x00 \\
 0x8 & 0x00 \\
 0x9 & 0x00 \\
 0xA & 0x00 \\
 0xB & 0x00 \\
 0xC & 0x02 \#valor SUB \\
 0xD & 0x06 \#valor ADD \\
 0xF & 0x03 \#valor LDA \\
 \hline
\end{tabular}
\label{tab-p2}
\end{table}


O programa 1 na tabela \ref{tab-p1} testa a instrução LDA e a instrução OUT.
O programa carrega o registrador A com o valor no endereço 0x0F e em seguida executa a instrução OUT.
O comportamento esperado é que o valor no registrador O seja o mesmo valor contido no endereço 0x0F.

O programa 2 na tabela \ref{tab-p2} testa a instrução as instruções que restam, SUM e SUB.
Primeiramente o programa executa a instrução LDA, e carrega o registrador A com o valor $3$.
Em seguida o programa executa a instrução ADD e soma o valor contido no endereço 0x0E (0x06) ao valor em A.
Finalmente, o programa executa SUB e subtrai o valor em A com o valor no endereço 0x0D (0x02). 
O programa encerra com a instrução OUT.
Espera-se que o valor contido no registrador O seja 0x07 ($3+6-2$).

O processador foi considerado correto quando o valor obtido da execução de cada programa concordava com o valor esperado.

