\section{CONSIDERAÇÕES FINAIS}

O objetivo desse trabalho foi expor alguns conceitos por traz do paradigma funcional e demonstrar como esses conceitos são úteis através da construção de um módulo de processamento de expressoes regulares.
O modulo foi implementado usando a linguagem Haskell, uma linguagem funcional.

Primeiramente, foi explicado o que são regexes do ponto de vista de um usuário e quais problemas elas resolvem.
Em seguida foi explicado, a partir de exemplo, como o paradgima funcional difere do paradigma imperativo.
Finalmente, foram introduzidas automatas, sua descrição formal, o algoritmo para executar uma automata e como regexes são equivalentes a automatas.

A partir desses conceitos foi explicado o método a ser utilizado nesse trabalho e um pouco mais sobre o objetivo.
O trabalho tem o foco de introduzir o paradigma funcional a partir de um problema real (processamento de regex).
Foi dado um exemplo de como pensar sobre um problema de maneira funcional, a partir de funções e como criar sequencias de funções para transformar a entrada no produto final.
Após isso foi feita a análise do código fonte, escrito de maneira funcional, onde foram introduzidas ferramentas referentes a esse paradigma.
Essa abordagem permitiu expor os pontos fracos e fortes do paradigma e também como fazer o que essas ferramentas fazem em uma linguagem imperativa.

Em conclusão, esse trabalho tem o objetivo de mostrar um mundo diferente da programação, um mundo que vem sido encorporado às linguagens imperativas, mesmo que muitos programadores desconhecam suas origens.


