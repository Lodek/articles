\section{EXPRESSÕES REGULARES E PROGRAMAÇÃO FUNCIONAL}
Como foi abordado na introdução, esse trabalho une dois temas: expressões regulares e programação funcional.
Esses temas serão, primeiramente, discutidos separadamente, e em seguida será feito uma previa de como será feita a construção do motor de processamento de expressões regulares.

\subsection{Introdução as expressões regulares}

Expressões regulares, tambem conhecidas como \emph{regex} (da junção do nome em inglês, \emph{regular expression}) são utilizadas para realizar buscas complexas sobre strings.
Para Cox,
"expressões regulares são uma notação que descreve um conjunto de strings.
Quando alguma string está no conjunto associado à expressão regular, pode-se dizer que essa expressão regular corresponde a esse string." \cite{cox}.

As regexes são utilizadas frequentemente, tanto para extrair informações que seguem um padrão ou para realizar buscas mais flexíveis ou parciais.
Como exemplo, suponha o problema de extrair todas os strings que correspondem a um horario em um texto.
A escrita de um horario segue uma estrutura padrão, HH:MM:SS onde HH delimita as horas, MM delimita os minutos e SS delimita os segundos.
Sem ter que construir todas as possíveis combinações de horas que seguem esse formato, uma simples varredura de texto é incapaz de extrair essa informação.
Esse problema pode ser resolvido tranquilamente usando regexes.

Uma expressão regular que realiza esta busca é,
\begin{equation}
  [0-9]\{2\}:[0-9]\{2\}:[0-9]\{2\} .
\end{equation}

Em palavras, os símbolos [0-9] representa qualquer carácter numerico entre 0 e 9.
De maneira geral os símbolos [] representam um conunto de caracteres \cite{python-re}.
O token \{2\} indica uma repetição, sendo equivalente à regex [0-9][0-9], ou seja dois carácteres numericos.
Os simbolos \{\} são usados para representar repetição \cite{python-re}.
O caractere ":" é interpretado de maneira literal.
Fazendo a união, a regex acima equivale a qualquer string que tenha o formato DD:DD:DD onde D indica qualquer digito de 0-9.
Podemos ver que esse formato é exatamente o formato definido anteriormente.

É importante ressaltar que existem inumeras variações e implementações de regexes, onde existem diferentes meta-caracteres para descrever operações.
Em \cite{mastering}, o autor discute as diferenças em regex entre as linguagens: PHP, .NET, Java e Perl.
Na documentação oficial da linguagem Python \cite{python-re} é dito que o dialeto usado é basedo nas expressões regulares da linguagem Perl com alguns adicionais.
Usuários UNIX também estão familiarizados com os \emph{wildcards} presentes nos shells, uma forma de regex.
Em resumo, existem diversos dialetos porém o objetivo das regexes não se altera, buscar por padrões.
A regex acima e todas as regexes subsequentes nesse texto serão escritos no dialeto da linguagem Perl. 

\subsection{Programação Funcional}

Essa seção aborda programação funcional e suas caracteristicas de maneira resumida.
Há muito a se falar sobre esse assunto pois ele é extensso e tem uma longa história.
Para abordar a programação funcional será tomado um foco que toma como base a computação.

De maneira geral, programas de computadores existem para resolver problemas computacionais.
Segundo \cite{matrix} um problema computacional é
" [...] uma especificação de entrada-saída que um procedimento tenha que satisfazer."
e um procedimento é
"[...] uma descrição precisa de uma computação; ele aceita entradas (chamadas de argumentos) e produz uma saída (chamado de valor de retorno).".
Ou seja, independente do paradigma utilizado para resolver o problema (funcional ou imperativa), ambos são capazes de definir um procedimento para um problema computacional, a grande diferença está em como esse procedimento é definido.

Segundo \cite{Bird},
\begin{quotation}
"Programção funcional é: um método para construção de programas que enfatiza funções e suas aplicações ao invéz de commandos e suas execuções; programação funcional faz uso de notações matemática simples que permite que problemas sejam descritos de maneira clara e concisa. [...]".
\end{quotation}
A programação imperativa foca em passos para resolver um problema, cada passo desse pode ser traduzido de maneira rasoavelmente direta em instruções de uma CPU.
Isso faz com que o procedimeto escrito imperativemente reflita muito mais a máquina do que ao homem.
O paradigma funcional tira o foco nos passos individuais para solucionar o problema e enfatiza uma estrutura para resolver o problema.

Em seguida serão abordados aspectos mais técnicos da programação funcional.

\subsubsection{Funções para resolver problemas}

Como visto anteriormente, a programação funcional propõe que problemas computacionais sejam resolvidos de maneira mais declarativa.
O foco muda de "quais passos é preciso para resolver esse problema" para "quais transformações aplicar nas minhas entradas para produzir a saída".

Para exemplificar essa idea considere o seguinte problema.
Dado uma lista de nomes, com nome, nome do meio e sobrenomes, crie uma lista com todas as combinações de primeiro nome e ultímo nome, ignorando nomes do meio e aonde as combinações cuja soma do primeiro e ultimo nome não exceda 15 caracteres (incluindo o espaço).
Para isso, ao invez de analisar os passos para processar esses dados, uma boa ideia é pensar em como manipular os dados para se obter o resultado desejado.
Para esse problema, sugere-se a seguinte solução:

\begin{enumerate}
\item{Separar cada nome da lista de nomes nos espaços e armazenar os nomes uma lista.}
\item{Filtrar listas que só possuem um elemento (somente um nome).}
\item{Filtrar listas e remover nomes do meio.}
\item{Criar uma lista de nomes e uma lista sobrenomes.}
\item{Realizar o produto cartesiano sobre essa lista e gerar uma lista de tuplas.}
\item{Tranformar tuplas em strings fazendo a concatenação do primeiro nome e do sobrenome.}
\end{enumerate}

Percebe-se que cada passo acima realiza uma única ação, sendo ela simples e clara e é interessante modelar cada um desses passos como uma função.
Na linguagem Haskell o tipos dos argumentos e do retorno de uma função é dado pela notação nomeDaFuncao :: arg1 -> arg2 -> ... -> retorno, onde arg1 e arg2 definem os tipos dos argumentos \cite{lipovaca}.
Para identificar listas em Haskell é usado o símbolo [], ou seja [Char] indica uma lista de caracteres e tuplas são indicatas com () onde (String, String) indica uma tupla com dois elementos, ambos strings.
Podemos agora reescrever o problema acima definindo todas as funções que serão utilizadas.

Primeiramente definiremos o problema enunciado como uma função usando a notação introduzida.
O problema inicial é a função $combinarNomes :: [String] -> [String]$ , ou seja uma função que recebe uma lista de Strings e retorna uma lista de Strings.
Em seguida, iremos declarar as funções que representam cada passo acima.

\begin{enumerate}
\item{separarNomes :: [String] -> [[String]]}
\item{tirarIncompleto :: [[String]] -> [[String]]}
\item{removerSobrenomes :: [[String]] -> [[String]]}
\item{gerarNomesESobrenomes :: [[String]] -> ([String], [String])}
\item{gerarCombinacoes :: ([String], [String]) -> [(String, String)]}
\item{concatenarNomes :: [(String, String)] -> [String]}
\end{enumerate}

Segundo \cite{lipovaca} a assinatura de uma função em Haskell combinado com um nome descritivo diz muito sobre a função e de fato, dados os nomes e sua assinatura, pode-se facilmente deduzir o que cada função está fazendo.

O objetivo dessa seção foi dar um exemplo alto nível de como é resolvido um problema de maneira funcional.

\subsection{Automatas e expressões regulares}

Como visto, as expressões regulares representam uma maneira conveniente de descrever conjutos de string.
Embora conveniente, a maneira na qual as expressões regulares foram introduzidas não permite uma tradução direta delas para um ambiente computacional.
Essa seção faz a ligação entre esses objetos teóricos e uma descrição matématica das mesmas.

\subsubsection{Definição de uma automata}

Segundo \cite{comp}, as automatas modelam um computador com uma quantidade minúscula de memória.
A ideia central de uma automata é representar uma estrutura computacional a partir de um conjunto de estados e entradas.

Os estados da automata constitui um conjunto denominado de $Q$, o conjunto de estados.
Dentre esses estados existe um único estado inicial da automata chamado de $q_o \in Q$.
Automatas recebem entradas a partir de simbolos, o conjuto de todos os símbolos reconhecidos por uma automata define um conjunto $\Sigma$ chamado de alfabeto da automata.
Os estados da automata podem ou não estar conectados, quando existe uma conexão entre dois estados essa conexão é representada por um simbolo $\alpha \in \Sigma$.
As transições entre estados de uma automata é representado por uma função $\delta$ onde $\delta : Q \times \Sigma \mapsto Q$, ou seja $\delta$ recebe dois argumentos, um estado e um símbolo e mapeia esse par a um estado.
Finalmente, a automata possui um conjunto de estados de aceitação $F$, onde $F \in Q$, caso a automata termine sua execução em um estado $q \in F$, o string de entrada foi aceito pela automata.
Formalmente, então, uma automata é uma tupla com 5 elementos $(Q, \Sigma, \delta, q_o, F)$ \cite{comp}.

A partir da descrição formal de uma automata, podemos definir uma rotina de computação.
De maneira breve, o objetivo dessa rotina é verificar que após processar o string de entrada a automata se encontra em um estado de aceitação.

Como foi visto, uma automata pode receber um conjunto de entradas que definem seu alfabeto $\Sigma$.
Deseja-se definir um procedimento onde dado um string de entrada e uma automata no seu estado inicial, retorne o estado final após processar a entrada.
Esse procedimento é definido como dado uma entrada $w=w_1w_2...w_n \mid w_i \in \Sigma$ e uma automata $M$ no seu estado inicial, será retornado um estado $q \in Q$.
Caso o estado final seja um estado de aceitação ($q \in F$), é dito que $M$ aceita $w$ \cite{comp}.
Formalmente, segundo \cite{comp} $M$ aceita $w = w_1w_2...w_n$ se: existe uma sequencia de estados $r_0, r_1, ... r_n \in Q$ se $r_0 = q_0$; $\delta(r_i, w_{i+1}) = r_{i+1},$ para $i = 0, ..., n-1$; $r_n \in F$.

O ponto chave desse discussão é apresentado por \cite{comp}, onde foi provado que é possível construir uma automata para qualquer regex.
Sendo assim, é possível definir padrões de busca usando uma expressão regular, converter essa expressão regulara para uma automata e usar essa automata para realizar a busca pelo padrão em um string.

\subsection{METODOLOGIA}

Como explicado na introdução, o foco destre trabalho é demonstrar alguns  elementos da programação funcional.
Para isso, foi escolhido o problema de implementar um módulo de busca de strings usando expressões regulares.
Será escolhido trechos de código especialmente interessante do módulo escrito que serão explciados a fundo.

Foi visto que uma expressão regular pode ser convertida em uma automata equivalente.
Sendo assim, o problema possui duas tarefas: criar submódulo para converter uma regex em uma automata e implementar um submódulo que permita criar e operar uma automata.
Serão definidas as arquiteturas de cada submódulo tal como o encadeamento de funções que serão chamadas para resolver cada problema, análogo ao que foi feito anteriormente.
Será explicado, de maneira alto nível, o que cada função faz baseada em suas entradas e saídas.
Isso irá motivar a introdução de tipos de dados únicos a programação funcional.

Além dessa inspeção de "caixa preta" das funções, os pontos principais do módulo será explicado em detalhe, o que permitirá aa análise de conceitos importantes no paradigma funcional.
Será feita uma comparação entre trechos escrito de maneira funcional e imperativa.
Essa comparação tem dois objetivos: introduzir conceitos referentes a linguagem funcional e identificar em quais situações um código funcional é mais simples, ou mais complexo, que o seu equivalente de maneira imperativa.
Para introduzir os conceitos do paradigma funcional, o código irá ser projetado tal que demonstre as diferentes ferramentas que compõe a caixa de ferramentas de um programador funcional.
As ferramentas simples abordarão conceitos comos imutalidade e recursão e as ferramentas mais complexas irão introduzir abstrações muito perculiares da programação funcional tal como \emph{Functors} e \emph{Applicative Functors}.
A metodologia escolhida tem como objetivo ser transparente quanto aos lados bons e ruins da programação funcional e também auxiliar a associação do paradigma imperativo ao funcional.
Dessa forma, um leitor familiar com programação imperativa poderá entender como um problema resolvido de maneira imperativa pode ser traduzido para um algoritimo funcional.

Em conclusão, o trabalho irá resolver o problema de criar um módulo de procura em texto usando expressões regulares.
O problema será quebrado em funções, exemplificando como resolver um problema a partir de funções ao invez de passos.
O código fonte do módulo criado será usado para introduzir conceitos sobre o paradigma funcional e familiarizar o leitor com algumas ferramentas.
Ao mesmo tempo, trechos de códigos funcionais serão comparados com seu equivalente escrito em uma linguagem imperativa, o que permitira associar conceitos imperativos a funcionais e expor os pontos fortes e fracos desse paradigma.

